% Der Befehl \newcommand kann auch benutzt werden um Variablen zu definieren:

% Semester:
\newcommand{\varSemester}{2. Semester}
% Fach:
\newcommand{\varFach}{Algorithmen und Datenstrukturen}
\newcommand{\varFachShort}{AuD}
% Teilgebiet:
%\newcommand{\varTeilGebiet}{Teilgebiet}
% Thema:
%\newcommand{\varThema}{Thema}
% Datum:
%\newcommand{\varDate}{\today}
% Dozent:
\newcommand{\varDozent}{Prof. Dr. rer. nat. Peter Jüttner}
% Autor:
\newcommand{\varAutor}{Christoph Stephan}
% E-Mail-Adresse des Dozenten:
\newcommand{\varEmailDozent}{peter.jüttner@th-deg.de}
% E-Mail-Adresse des Autors:
\newcommand{\varEmailAutor}{chm.stephan@outlook.com}
% E-Mail-Adresse anzeigen (true/false):
\newcommand{\varZeigeEmail}{true}
% Anhang anzeigen (true/false):
\newcommand{\varZeigeAnhang}{false}
% Literaturverzeichnis anzeigen (true/false):
\newcommand{\varZeigeLiteraturverzeichnis}{false}
% Stil der Einträge im Literaturverzeichnis
\newcommand{\varLiteraturLayout}{unsrtdin}
