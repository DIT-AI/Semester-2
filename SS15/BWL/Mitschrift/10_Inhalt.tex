\chapter{Einführung}
\section{Ökonomisches Prinzip}
\paragraph{Maximum-Prinzip:} Maximierung des Outputs bei gegebenem Input
\paragraph{Minimum-Prinzip:} Minimierung des Inputs bei gegebenem Output
\paragraph{Optimum-Prinzip:} Maximierung der Differenz von Input und Output

\paragraph{Definition Wirtschaften:} Disponieren über knappe Güter.\\
Planvolle zielgerichtete Tätigkeit knappe Güter bestmöglich nutzen.

\paragraph{Maslowsche Bedürfnispyramide}
\begin{compactitem}
	\item Materielle Bedürfnisse
	\item Immaterielle Bedürfnisse
\end{compactitem}

\imgw{Bilder/Kreislauf}{Wirtschaftskreislauf}{}{12cm}
\imgw{Bilder/Einheiten}{Wirtschaftseinheiten}{}{12cm}
\imgw{Bilder/Prozess}{Wertschöpfungsprozess}{}{13cm}

\clearpage
\paragraph{Marktformen}
\imgw{Bilder/Marktformen}{Marktformen}{}{13cm}

\paragraph{Preisbildung am Markt:} abhängig von Wettbewerber, Lieferanten, Kosten und Kunden

\imgw{Bilder/Ziele}{Ziele}{}{13cm}

\clearpage
\chapter{Investitionsrechnung}
\paragraph{Investitionsbegriff:} Investition, Herstellung und Erwerb von Sachgütern

\paragraph{Finanzierungsplan}\quad\\
\ul{Mittelverwendung/Aktiva:} wohin (Anlagevermögen, Umlaufvermögen)
\ul{Mittelherkunft/Passiva:} woher (Eigenkapital, Fremdkapital\red{?}, Rückstellungen, Verbindlichkeiten)

\paragraph{Investitionsarten:} Richtung wie der Geldbetrag angelegt wird.

\begin{compactitem}
	\item Sachinvestitionen (z.B. Grundstücke, Gebäude, Maschinen)
	\item Finanzinvestitionen (z.B. Aktien, Beteiligungen, Pfandbriefe)
	\item Immaterielle Investitionen (z.B. Forschung \& Entwicklung, Aus- \& Weiterbildung, Lizenzen)
\end{compactitem}

\clearpage
\section{Bewertung von Investitionsvorhaben}
\begin{tabular}{l|l}
	{\bf Statistische Verfahren} & {\bf Dynamische Verfahren}\\\hline
	Rentabilitätsrechnung & Barwertverfahren\\
	Kostenvergleichsrechnung & Annuitätenmethode\\
	Gewinnvergleichsmethode & Interner Zins\\
	Amortationsmethode & \\
\end{tabular}

\clearpage
\subsection{Aufzinsen und Abzinsen}
Ausgangspunkt der Investitionsplanung ist die Annahme, dass ein Geldbetrag, der {\flqq heute\frqq} fällig wird, anders zu beurteilen ist als ein gleicher Geldbetrag, der {\flqq morgen\frqq} fällig wird.

Daraus folgt -- Grundlage der Investitionstheorie ist das Auf- und Abzinsen von Zahlungsströmen, die zu verschiedenen Zeitpunkten anfallen.

\paragraph{Aufzinsen}
\redbox{$K_n=K_0\cdot(1+p)^n$}

\paragraph{Abzinsen}
\redbox{$K_0=\dfrac{K_n}{(1+p)^n}$}

\paragraph{Durchschnittszinsen}
\redbox{$p=\sqrt[n]{\dfrac{K_n}{K_0}}-1$}

\paragraph{Investitionszeit} um aus Anfangsbetrag Endbetrag zu machen
\redbox{$n=\dfrac{\log{\left(\dfrac{K_n}{K_0}\right)}}{\log{(1+p)}}$}

\clearpage
\subsubsection{Zusammenfassung}
\begin{compactitem}
	\item Zahlungen werden zeitabhängig bewertet.
	\item Vergleich von verschiedenen Zahlungen immer als hätte man das Kapital am Kapitalmarkt angelegt.
	\item Wesentliche Formel: Aufzinsformel $K_n=K_0\cdot(1+p)^n$
\end{compactitem}

\subsection{Kapitalwertmethode}
Die Kapitalwertmethode setzt voraus, dass der Investor weiß, welchen {\flqq Zinsgewinn\frqq} er aus einem Investitionsobjekt mindestens erwirtschaften will. Dieses vom Marktzins und Risikogesichtspunkten abhängige Mindestverzinsung nennt man Kalkulationszinssatz (p).

Die Kapitalwertmethode prüft, ob in einem Investitionsobjekt mindestens der Kalkulationszinssatz steckt und es sich damit lohnt.

\subsection{Statistische Methoden}
Statistische Methoden nutzen nicht den Zeitwert des Geldes.

\subsubsection{Rentabilitätsrechnung}
Kennzahl die den Projektgewinn in Verhältnis zum Kapitaleinsatz setzt. Gewinn und Kapital als durchschnittlich pro Periode.

Verhältnis der beiden Größen is die Rentabilität des Projekts (durchschnittliche Verzinsung des Projekts).

\greenbox{Liquidationserlös = Schrottwert}

\greenbox{durchschnittlich gebundenes Kapital = Kapitaleinsatz}

\greenbox{kalkulatorische Zinsen = Zinskosten}

\greenbox{UV-Erhöhung = Erhöhung des Umlaufvermögens}

\paragraph{Berechnungen:}\quad\\
\redbox{$\text{Kapitaleinsatz} = \dfrac{\text{Anschaffungswert} + \text{Liquidationserlös} + (\text{UV-Erhöhung})}{2}$} (Durchschnitt aus Anschaffungswert und Schrottwert)\\
\redbox{$\text{Abschreibungen} = \dfrac{\text{Anschaffungswert} - \text{Liquidationserlös} - (\text{UV-Erhöhung})}{\text{Nutzungsdauer}}$}\\
\redbox{$\text{kalkulatorische Zinsen} = \text{Zinsfuß}\cdot\text{Kapitaleinsatz}$}\\

\imgw{Bilder/Investitionsrechnung}{Formeln für Investitionsrechnung}{}{17cm}

\clearpage
\subsubsection{Kostenvergleichsrechnung}
Vergleich der Gesamtkosten zwischen verschiedenen Investitionen.\\
Eignet sich, wenn Nutzen der Varianten etwa gleich oder unbekannt.

\paragraph{Idee:} Die Kostenvergleichsrechnung ziel nur darauf ab die Kosten eines Projektes zu vergleichen

\paragraph{Auswahlkriterium:} Gewählt wird das Projekt mit den niedrigeren Kosten

\paragraph{Kritik:}
\begin{compactitem}
	\item Reines Vergleichsverfahren
	\item Keinen Zeitwert des Geldes
	\item Hohe Anforderungen an Vorbedingungen
\end{compactitem}

\paragraph{Jährliche Gesamtkosten:} $\text{Jährliche Betriebskosten} + \text{Jährliche Abschreibungen} + \text{Zinskosten}$

\clearpage
\subsubsection{Gewinnvergleichsmethode}
Vergleich des Gewinns mehrerer Investitionen.\\
Eignet sich, wenn Kapitaleinsatz und Nutzungsdauer etwa gleich.

\paragraph{Idee:} Reiner Vergleich der durchschnittlichen Erträge von zwei Projekten

\paragraph{Auswahlkriterium:}
\begin{compactitem}
	\item Projekt muss Gewinn erwirtschaften
	\item Gewählt wird das Projekt mit dem höheren Ertrag
\end{compactitem}

\paragraph{Kritik:} Keine Berücksichtigung des Kapitaleinsatzes

\paragraph{Jährlicher Gewinn:} $\text{Jährlicher Erlös} - \text{Jährliche Betriebskosten} - \text{Jährliche Abschreibungen} - \text{Zinskosten}$\\
$\Rightarrow$ $\text{Jährlicher Erlös} - \text{Jährliche Gesamtkosten}$

\clearpage
\subsubsection{Renditeberechnung ROI}
Berechnen der Bruttorendite der Investitionen in Prozent des $\varnothing$ Kapitaleinsatzes.\\
Eignet sich zur Beurteilung einzelner Investitionen oder zum Vergleich unterschiedlicher Investitionsvarianten (die nicht direkt miteinander verglichen werden können).

\paragraph{ROI [\%]:} $\dfrac{\text{Gewinn} + \text{Zinskosten}}{\varnothing \text{Kapitaleinsatz}}\cdot 100$

\clearpage
\subsubsection{Amortisationsrechnung (Pay-Back-Methode)}
Ermitteln der Zeit, die verstreicht, bis eine Investition durch ihre Rückflüsse (Cashflows) amortisiert ist.\\
Eignet sich für Überschlagsrechnungen und als Risikomaßstab.

\paragraph{Idee:} Berechnung des Zeitpunktes zu dem ein Projekt zurückgezahlt wurde

\paragraph{Bedingung:} Laufzeit vergleichbar
\paragraph{Methode:} Abziehen der Zuflüsse von den Investitionen des Projektes und Überprüfen, wann das Projekt positiv wird

\paragraph{Auswahlkriterium:}
\begin{compactitem}
	\item Ein Projekt wird gemacht, wenn es seine Investition zurückzahlt
	\item Gewählt wird das Projekt mit der schnelleren Rückzahlungszeit
\end{compactitem}

\paragraph{Kritik:}
\begin{compactitem}
	\item Keinen Zeitwert des Geldes
	\item Keine strategischen Entscheidungen
\end{compactitem}

\redbox{$\text{Wiedergewinnungszeit} = \dfrac{\text{Kapitaleinsatz}}{\text{Jährlicher Nutzen}}$}

\redbox{$\text{Rückflusszahl} = \dfrac{\text{Nutzungsdauer}}{\text{Wiedergewinnungszeit}}$}

\clearpage
\subsection{Dynamische Methoden}
\clearpage
\subsubsection{Interner Ertragssatz (IRR)}
\imgw{Bilder/IRR}{IRR}{}{16cm}

\clearpage
\subsubsection{Gegenwart-Methode (Kapitalwertmethode)}
\imgw{Bilder/GWM}{Gegenwart-Methode}{}{16cm}

\clearpage
\subsubsection{Annuitätenmethode (Kapitalwertmethode)}
\imgw{Bilder/AM}{Annuitätenmethode}{}{16cm}

\section{Abgeleitete Größen}
\section{Nominal und effektiv Zins}
\section{Skonto}
\section{Rentenbegriff}
\section{Barwert und Endwert}
\section{Rückverteilungsfaktoren}