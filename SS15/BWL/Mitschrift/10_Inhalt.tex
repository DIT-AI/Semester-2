\chapter{Einführung}
\section{Ökonomisches Prinzip}
\paragraph{Maximum-Prinzip:} Maximierung des Outputs bei gegebenem Input
\paragraph{Minimum-Prinzip:} Minimierung des Inputs bei gegebenem Output
\paragraph{Optimum-Prinzip:} Maximierung der Differenz von Input und Output

\paragraph{Definition Wirtschaften:} Disponieren über knappe Güter.\\
Planvolle zielgerichtete Tätigkeit knappe Güter bestmöglich nutzen.

\paragraph{Definition Betrieb:} planvoll organisierte Wirtschaftseinheit, in der Sachgüter und Dienstleistungen erstellt und abgesetzt werden.

\paragraph{Maslowsche Bedürfnispyramide}
\begin{compactitem}
	\item Materielle Bedürfnisse
	\item Immaterielle Bedürfnisse
\end{compactitem}
\begin{compactitem}
	\item Individual-Bedürfnisse
	\item Kollektiv-Bedürfnisse
\end{compactitem}
\begin{compactitem}
	\item Existenz-Bedürfnisse (primär)
	\item Kultur-Bedürfnisse (sekundär)
	\item Luxus-Bedürfnisse (tertiär)
\end{compactitem}

\imgw{Bilder/MaslowscheBeduerfnispyramide}{Maslowsche-Bedürfnispyramide}{}{13cm}

\imgw{Bilder/Kreislauf}{Wirtschaftskreislauf}{}{12cm}
\imgw{Bilder/Einheiten}{Wirtschaftseinheiten}{}{12cm}
\imgw{Bilder/Prozess}{Wertschöpfungsprozess}{}{13cm}

\clearpage
\paragraph{Marktformen}
\imgw{Bilder/Marktformen}{Marktformen}{}{13cm}

\paragraph{Preisbildung am Markt:} abhängig von Wettbewerber, Lieferanten, Kosten und Kunden

\imgw{Bilder/Ziele}{Ziele}{}{13cm}

\clearpage
\chapter{Investitionsrechnung}
\paragraph{Investitionsbegriff:} Investition, Herstellung und Erwerb von Sachgütern

\paragraph{Finanzierungsplan}\quad\\
\ul{Mittelverwendung/Aktiva:} wohin (Anlagevermögen, Umlaufvermögen)\\
\ul{Mittelherkunft/Passiva:} woher (Eigenkapital, Fremdkapital, Rückstellungen, Verbindlichkeiten)

\paragraph{Investitionsarten:} Richtung wie der Geldbetrag angelegt wird.

\begin{compactitem}
	\item Sachinvestitionen (z.B. Grundstücke, Gebäude, Maschinen)
	\item Finanzinvestitionen (z.B. Aktien, Beteiligungen, Pfandbriefe)
	\item Immaterielle Investitionen (z.B. Forschung \& Entwicklung, Aus- \& Weiterbildung, Lizenzen)
\end{compactitem}

\clearpage
\section{Bewertung von Investitionsvorhaben}
\begin{tabular}{l|l}
	{\bf Statistische Verfahren} & {\bf Dynamische Verfahren}\\\hline
	Rentabilitätsrechnung & Barwertverfahren\\
	Kostenvergleichsrechnung & Annuitätenmethode\\
	Gewinnvergleichsmethode & Interner Zins\\
	Amortationsmethode & \\
\end{tabular}

\clearpage
\subsection{Aufzinsen und Abzinsen}
Ausgangspunkt der Investitionsplanung ist die Annahme, dass ein Geldbetrag, der {\flqq heute\frqq} fällig wird, anders zu beurteilen ist als ein gleicher Geldbetrag, der {\flqq morgen\frqq} fällig wird.

Daraus folgt -- Grundlage der Investitionstheorie ist das Auf- und Abzinsen von Zahlungsströmen, die zu verschiedenen Zeitpunkten anfallen.

\paragraph{Aufzinsen}
\redbox{$C_n=C_0\cdot(1+i)^n$}

\paragraph{Abzinsen}
\redbox{$C_0=\dfrac{C_n}{(1+i)^n}$}

\paragraph{Durchschnittszinsen}
\redbox{$i=\sqrt[n]{\dfrac{C_n}{C_0}}-1$}

\paragraph{Investitionszeit} um aus Anfangsbetrag Endbetrag zu machen
\redbox{$n=\dfrac{\log{\left(\dfrac{C_n}{C_0}\right)}}{\log{(1+i)}}$}

\clearpage
\subsubsection{Zusammenfassung}
\begin{compactitem}
	\item Zahlungen werden zeitabhängig bewertet.
	\item Vergleich von verschiedenen Zahlungen immer als hätte man das Kapital am Kapitalmarkt angelegt.
	\item Wesentliche Formel: Aufzinsformel $C_n=C_0\cdot(1+i)^n$
\end{compactitem}

\subsection{Kapitalwertmethode}
Die Kapitalwertmethode setzt voraus, dass der Investor weiß, welchen {\flqq Zinsgewinn\frqq} er aus einem Investitionsobjekt mindestens erwirtschaften will. Dieses vom Marktzins und Risikogesichtspunkten abhängige Mindestverzinsung nennt man Kalkulationszinssatz (p).

Die Kapitalwertmethode prüft, ob in einem Investitionsobjekt mindestens der Kalkulationszinssatz steckt und es sich damit lohnt.

\subsection{Statistische Methoden}
Statistische Methoden nutzen nicht den Zeitwert des Geldes.

\subsubsection{Rentabilitätsrechnung}
Kennzahl die den Projektgewinn in Verhältnis zum Kapitaleinsatz setzt. Gewinn und Kapital als durchschnittlich pro Periode.

Verhältnis der beiden Größen is die Rentabilität des Projekts (durchschnittliche Verzinsung des Projekts).

\greenbox{Liquidationserlös = Schrottwert}

\greenbox{durchschnittlich gebundenes Kapital = Kapitaleinsatz}

\greenbox{kalkulatorische Zinsen = Zinskosten}

\greenbox{UV-Erhöhung = Erhöhung des Umlaufvermögens}

\paragraph{Berechnungen:}\quad\\
\redbox{$\text{Kapitaleinsatz} = \dfrac{\text{Anschaffungswert} + \text{Liquidationserlös} + (\text{UV-Erhöhung})}{2}$} (Durchschnitt aus Anschaffungswert und Schrottwert)\\
\redbox{$\text{Abschreibungen} = \dfrac{\text{Anschaffungswert} - \text{Liquidationserlös} - (\text{UV-Erhöhung})}{\text{Nutzungsdauer}}$}\\
\redbox{$\text{kalkulatorische Zinsen} = \text{Zinsfuß}\cdot\text{Kapitaleinsatz}$}\\

\imgw{Bilder/Investitionsrechnung}{Formeln für Investitionsrechnung}{}{17cm}

\clearpage
\subsubsection{Kostenvergleichsrechnung}
Vergleich der Gesamtkosten zwischen verschiedenen Investitionen.\\
Eignet sich, wenn Nutzen der Varianten etwa gleich oder unbekannt.

\paragraph{Idee:} Die Kostenvergleichsrechnung ziel nur darauf ab die Kosten eines Projektes zu vergleichen

\paragraph{Auswahlkriterium:} Gewählt wird das Projekt mit den niedrigeren Kosten

\paragraph{Kritik:}
\begin{compactitem}
	\item Reines Vergleichsverfahren
	\item Keinen Zeitwert des Geldes
	\item Hohe Anforderungen an Vorbedingungen
\end{compactitem}

\paragraph{Jährliche Gesamtkosten:} $\text{Jährliche Betriebskosten} + \text{Jährliche Abschreibungen} + \text{Zinskosten}$

\clearpage
\subsubsection{Gewinnvergleichsmethode}
Vergleich des Gewinns mehrerer Investitionen.\\
Eignet sich, wenn Kapitaleinsatz und Nutzungsdauer etwa gleich.

\paragraph{Idee:} Reiner Vergleich der durchschnittlichen Erträge von zwei Projekten

\paragraph{Auswahlkriterium:}
\begin{compactitem}
	\item Projekt muss Gewinn erwirtschaften
	\item Gewählt wird das Projekt mit dem höheren Ertrag
\end{compactitem}

\paragraph{Kritik:} Keine Berücksichtigung des Kapitaleinsatzes

\paragraph{Jährlicher Gewinn:} $\text{Jährlicher Erlös} - \text{Jährliche Betriebskosten} - \text{Jährliche Abschreibungen} - \text{Zinskosten}$\\
$\Rightarrow$ $\text{Jährlicher Erlös} - \text{Jährliche Gesamtkosten}$

\clearpage
\subsubsection{Renditeberechnung ROI}
Berechnen der Bruttorendite der Investitionen in Prozent des $\varnothing$ Kapitaleinsatzes.\\
Eignet sich zur Beurteilung einzelner Investitionen oder zum Vergleich unterschiedlicher Investitionsvarianten (die nicht direkt miteinander verglichen werden können).

\paragraph{ROI [\%]:} $\dfrac{\text{Gewinn} + \text{Zinskosten}}{\varnothing \text{Kapitaleinsatz}}\cdot 100$

\clearpage
\subsubsection{Amortisationsrechnung (Pay-Back-Methode)}
Ermitteln der Zeit, die verstreicht, bis eine Investition durch ihre Rückflüsse (Cashflows) amortisiert ist.\\
Eignet sich für Überschlagsrechnungen und als Risikomaßstab.

\paragraph{Idee:} Berechnung des Zeitpunktes zu dem ein Projekt zurückgezahlt wurde

\paragraph{Bedingung:} Laufzeit vergleichbar
\paragraph{Methode:} Abziehen der Zuflüsse von den Investitionen des Projektes und Überprüfen, wann das Projekt positiv wird

\paragraph{Auswahlkriterium:}
\begin{compactitem}
	\item Ein Projekt wird gemacht, wenn es seine Investition zurückzahlt
	\item Gewählt wird das Projekt mit der schnelleren Rückzahlungszeit
\end{compactitem}

\paragraph{Kritik:}
\begin{compactitem}
	\item Keinen Zeitwert des Geldes
	\item Keine strategischen Entscheidungen
\end{compactitem}

\redbox{$\text{Wiedergewinnungszeit} = \dfrac{\text{Kapitaleinsatz}}{\text{Jährlicher Nutzen}}$}

\redbox{$\text{Rückflusszahl} = \dfrac{\text{Nutzungsdauer}}{\text{Wiedergewinnungszeit}}$}

\clearpage
\subsection{Dynamische Methoden}
\clearpage
\subsubsection{Interner Ertragssatz (IRR)}
\imgw{Bilder/IRR}{IRR}{}{16cm}

\clearpage
\subsubsection{Gegenwart-Methode (Kapitalwertmethode)}
\imgw{Bilder/GWM}{Gegenwart-Methode}{}{16cm}

\clearpage
\subsubsection{Annuitätenmethode (Kapitalwertmethode)}
\imgw{Bilder/AM}{Annuitätenmethode}{}{16cm}

\clearpage
\section{Renten}
\paragraph{Definition:} Renten sind gleichförmige,
äquidistante Zahlungsreihen.

\clearpage
\subsection{Barwert}
\redbox{$q=1+i$}\\
\redbox{$\text{BWF} = \dfrac{1-q^{-n}}{q-1} = \dfrac{q^n-1}{iq^n}$}\\
\redbox{$\text{Barwert} = C\cdot\text{BWF} = C\cdot\dfrac{q^n-1}{iq^n}$}

\clearpage
\subsection{Endwert}
\redbox{$q=1+i$}\\
\redbox{$\text{EWF} = \dfrac{1-q^n}{1-q} = \dfrac{q^n-1}{i}$}\\
\redbox{$\text{Endwert} = C\cdot\text{EWF} = C\cdot\dfrac{q^n-1}{i}$}

\bluebox{$\text{BWF} = \text{EWF}\cdot\dfrac{1}{q^n}$}

\clearpage
\subsection{Kapitalwiedergewinnungsfaktor}
\redbox{$\text{KWF} = \dfrac{1}{\text{BWF}}$}\\
\redbox{$C = \text{KWF}\cdot\text{Barwert}$}

\clearpage
\subsection{Rückverteilungsfaktor}
\redbox{$\text{RVF} = \dfrac{1}{\text{EWF}}$}\\
\redbox{$C = \text{RVF}\cdot\text{Endwert}$}

\clearpage
\section{Optimaler Ersatzzeitpunkt}
Weiternutzung einer Anlage im Jahr $n+1$ ist sinnvoll wenn\\
\redbox{$\text{Restwert}_n\cdot(1+i)<\text{Nettoeinkünfte}_{n+1}+\text{Restwert}_{n+1}$}\\
d.h. es ist sinnvoll, wenn die Einkünfte durch die Weiternutzung der Anlage und der Restwert am Ende des nächsten Jahres noch immer größer sind als der aktuelle Restwert inklusive Verzinsung für ein Jahr.

Daraus ergibt sich der Grenzgewinn:\\
\redbox{$\text{Grenzgewinn}_{n+1} = \text{Nettoeinkünfte}_{n+1}+\text{Restwert}_{n+1} - \text{Restwert}_n\cdot(1+i)$}\\
$\text{Grenzgewinn}_{n+1} > 0$: Weiterverwendung in Periode $n+1$\\
$\text{Grenzgewinn}_{n+1} < 0$: Liquidation am Ende der Periode $n$

Unterberücksichtigung, dass bei keiner Weiterverwendung eine neue Anlage gekauft werden muss. Diese ist für $x$ Jahre vorgesehen und kostet einmalig $A_0$. Daraus ergibt sich der zeitliche Wiedergewinn der nötig ist damit sich die Anlage abzahlt.
Dies muss dann noch vom Gewinn im nächsten ($n+1$) Jahr abgezogen werden. Ist dieser zeitliche Durchschnittsgewinn immer noch größer als der zeitliche Grenzgewinn, sollte die Anlage ersetzt werden durch eine neue.\\
\redbox{$\text{zeitlicher Durchschnittsgewinn}_{n+1} = \text{Nettoeinkünfte mit neuer Anlage}_{n+1} - A_0 \cdot \text{KWF}$}\\
$\text{KWF} = \dfrac{iq^x}{q^x-1}$\\
$\text{Grenzgewinn}_{n+1} > \text{zeitlicher Durchschnittsgewinn}_{n+1}$: Weiterverwendung in Periode $n+1$\\
$\text{Grenzgewinn}_{n+1} < \text{zeitlicher Durchschnittsgewinn}_{n+1}$: Liquidation am Ende der Periode $n$

\clearpage
\chapter{Finanzierung}
Finanzierung heißt Geldmittelbeschaffung (Kapitalbeschaffung).\\
Investition steht der Finanzierung gegenüber (Kapitalverwendung).

Bilanz gibt Auskunft über Vermögen.

\paragraph{Investition:} Herstellung und Erwerb von Sachgütern des Anlagevermögens. (Allgemein: Bilanzpassiva in Bilanzaktiva)

\clearpage
\section{Kapital}
\begin{compactitem}
	\item {\bf abstraktes Kapital:} Passivseite der Bilanz: Eigenkapital und Fremdkapital (Geldmittelherkunft, wie wurde das Vermögen finanziert)
	\item {\bf konkretes Kapital:} Aktivseite der Bilanz: Geld, Rechte und Sachgüter (Geldmittelverwendung, Vermögen im Unternehmen)
\end{compactitem}

\paragraph{Kapitalbeschaffung:} Bereitstellen von finanziellen Mitteln jeder Art, d.h.
\begin{compactitem}
	\item zur Durchführung der Leistungserstellung und Leistungsverwertung
	\item zur Vornahme bestimmter außerordentlicher finanztechnischer Vorgänge (Unternehmensgründung, Kapitalerhöhung, Fusion, Sanierung, Liquidation, etc.)
\end{compactitem}

\clearpage
\subsection{Aktiva-Seite}\quad\\
Anlagevermögen:
\begin{compactitem}
	\item Lizenzen / Patente
	\item Grundstücke
	\item Immobilien
	\item Maschinen
	\item Kraftfahrzeuge
	\item Betriebs- und Geschäftsausstattung
\end{compactitem}

Umlaufvermögen:
\begin{compactitem}
	\item liquiden Mitteln (Bank, Kasse)
	\item Forderungen aus Lieferungen und Leistungen
	\item Vorratsvermögen (Roh-, Hilfs- und Betriebsstoffe)
	\item Lagervermögen (fertige und halbfertige Erzeugnisse)
\end{compactitem}

\clearpage
\subsection{Passiva-Seite}
Diese Seite der Bilanz gibt an, woher die Geldmittel beschafft werden. Es wird unterschieden zwischen Eigenkapital und Fremdkapital. Jedoch gibt es bei der Finanzierung vielfältigere Möglichkeiten.

\clearpage
\section{Finanzierungsarten}
\paragraph{Unterscheidung in:} Außen- und Innenfinanzierung sowie Eigen- und Fremdfinanzierung

\imgw{Bilder/Finanzierungsarten}{Finanzierungsarten}{}{13cm}

\clearpage
\subsection{Selbstfinanzierung als Innenfinanzierung}
\begin{compactitem}
	\item aus Abschreibungen (als Ausgaben in GuV-Rechnung)
	\item aus Rückbehaltung von Gewinnen
	\begin{compactitem}
		\item Offene Selbstfinanzierung (Einbehalt von ausgewiesenen, versteuerten Gewinnen)
		\begin{compactitem}
			\item Verzicht auf Entnahme (Personengesellschaft), Gewinnausschüttung (Kapitalgesellschaft)
			\item Nachteil: Gewinn zu 50\% versteuern
		\end{compactitem}
		\item Stille Selbstfinanzierung (Einbehalt von nicht ausgewiesenen Gewinnen, durch bilanzpolitische Maßnahmen)
		\begin{compactitem}
			\item Unterbewertung des Vermögens
			\item Überbewertung der Passiva
		\end{compactitem}
	\end{compactitem}
\end{compactitem}

\clearpage
\subsection{Kapitalbedarf}
\begin{compactitem}
	\item {\bf positiv} wenn Auszahlungsüberschuss
	\item {\bf negativ} wenn Einzahlungsüberschuss
\end{compactitem}

Deckung mit Eigenkapital, Fremdkapital, Einzahlungsüberschüsse (negativer Kapitalbedarf)

\clearpage
\subsection{Finanzierung durch negativen Kapitalbedarf als Innenfinanzierung}
Einzahlungen, die ein Unternehmen aus dem Umsatzprozess erzielt, können sich aus Rückflüssen und Überschüssen zusammensetzen.

\paragraph{Rückflüsse}
Erstattungen für Vorleistungen die selbst von außen bezogen wurden (gelieferte Rohstoffe, Waren, Abschreibung (Wertminderung) der gekauften Maschinen)

\paragraph{Überschüsse}
wenn Einzahlungen $>$ Vorleistungen

\begin{compactitem}
	\item Rückflussfinanzierung
	\begin{compactitem}
		\item mit Abschreibungsgegenwerten
	\end{compactitem}
	\item Finanzierung mit Eigenkapital
	\begin{compactitem}
		\item eigenes Geld
		\item Eigenkapital von Geschäftspartnern
		\item Risikokapital
	\end{compactitem}
	\item Finanzierung mit Fremdkapital
	\begin{compactitem}
		\item Über Bank/öffentliche Förderkredite\\
		$\Rightarrow$ wollen Sicherheit: Bürgschaften, Vermögensübertragung
		\item frei verwendbar
		\item mit Zinsen belegt
		\item Bank gewährt Kredit für Vermögenswerte eher als einen für Geschäftsunterhalt\\
		$\Rightarrow$ Produktionsunternehmen meist kreditwürdiger als Dienstleistungsunternehmen (Maschinen können bei Insolvenz verkauft werden)
	\end{compactitem}
\end{compactitem}

\paragraph{Finanzierungsregel}
\begin{compactitem}
	\item langfristige Finanzierung für langfristige Investitionen
	\item kurzfristige Finanzierung für kurzfristigen Geldbedarf
\end{compactitem}

\clearpage
\section{Finanzierungsquellen}
\begin{tabular}{l|l}
	{\bf Eigenmittel} & {\bf Fremdmittel}\\\hline
	Eigenkapitalhilfe & Öffentliche Förderdarlehen\\
	Ersparnisse & Staatliche Zuschüsse\\
	Sachmitteleinlagen & Sonstige Darlehen\\
	Kapitalanlagen & Leasing\\
\end{tabular}
\begin{compactitem}
	\item Frühzeitige Sichtung von Kapitalquellen
	\item Vergleich verschiedener Kapitalquellen
	\item Zinsbelastung berechnen
\end{compactitem}

\clearpage
\subsection{Dynamischer Prozess - Kapitalbedarf}
\begin{compactitem}
	\item Pre/Seed-Phase
	\begin{compactitem}
		\item Erforschung, Entwicklung
	\end{compactitem}
	\item Early-Stage-Phase
	\begin{compactitem}
		\item Forschung, Produktkonzeption, Unternehmensgründung, Aufnahme der operativen Geschäftstätigkeit
	\end{compactitem}
	\item Later-Stage/Expansions-Stage Phase
	\begin{compactitem}
		\item Finanzierung des Wachstums, Ausweitung und Ausbau, Erreichen des Break-Even-Punktes
	\end{compactitem}
\end{compactitem}

\clearpage
\subsection{Investitionsplanung}
\paragraph{Arten}
\begin{compactitem}
	\item Einzelentscheidung (machen oder nicht?)
	\item Auswahlentscheidung (was ist besser?)
	\item Bestimmung der optimalen Nutzungsdauer (Ersatzzeitpunkt?)
\end{compactitem}

\paragraph{Zwecke}
\begin{compactitem}
	\item Sachinvestitionen
	\item Finanzinvestitionen
	\item Investitionen im Personalbereich
	\item Investitionen in Forschung und Entwicklung etc.
	\item Ersatzinvestitionen
	\item Erweiterungsinvestitionen
\end{compactitem}

\imgw{Bilder/Finanzierungsplan}{Finanzierungsplan}{}{15.5cm}

\clearpage
\subsection{Investitions- und Finanzierungsphasen}
\imgw{Bilder/Phasen1}{Phasen}{}{16cm}
\imgw{Bilder/Phasen2}{Phasen}{}{16cm}
\imgw{Bilder/Phasen3}{Phasen}{}{16cm}
\imgw{Bilder/Phasen4}{Phasen}{}{16cm}

\clearpage
\subsection{Systematisierung von Finanzierungsmöglichkeiten}
\begin{compactitem}
	\item Herkunft des Kapitals (Außen- und Innenfinanzierung)
	\item Rechtsstellung der Kapitalgeber (Eigenfinanzierung und Fremdfinanzierung)
	\item Dauer der Kapitalbereitstellung (unbefristet, langfristig, mittelfristig, kurzfristig)
	\item Anlass der Finanzierung (Gründung, Kapitalerhöhung, Fusion, Umwandlung)
\end{compactitem}

\imgw{Bilder/FinanzierungsartenVergleich}{Finanzierungsarten im Vergleich}{}{16cm}

\clearpage
\subsection{Gesamtübersicht Finanzierungsarten}
\begin{tabular}{|L{5cm}|L{4cm}|L{6.2cm}|}
	\hline
	{\bf Rechtstellung/\linebreak Kapitalhaftung} & {\bf Außenfinanzierung} & {\bf Innenfinanzierung}\\\hline
	Eigenfinanzierung & Beteiligungs-/\linebreak Einlagenfinanzierung & Selbstfinanzierung\\\hline
	Mischformen\linebreak Eigenkapital/Fremdkapital & (Mezzanine) & Finanzierung aus Abschreibungen\linebreak Finanzierung aus sonstigen Kapitalfreisetzungen\\\hline
	Fremdfinanzierung & Kreditfinanzierung & Finanzierung aus Rückstellungsgegenwerten\\\hline
\end{tabular}

\clearpage
\chapter{Effektiver Jahreszins}
\paragraph{Skonto} ist Kredit von Lieferanten für den Zeitraum des Zahlungsaufschubs.

Ob sich die Beanspruchung von Skonto lohnt entscheidet der \ul{effektive (tatsächliche) Jahreszins}.

Zum Vergleich steht der Bankzins, den man z.B. für einen kurzfristigen Kredit in Anspruch nehmen müsste, damit man innerhalb der Skontofrist die Rechnung bezahlen kann.

\redbox{$\text{effektiver Jahreszins} = \dfrac{100}{100 - \text{Skontosatz}}\cdot\dfrac{360\cdot\text{Skontosatz}}{\text{Zahlungsziel} - \text{Skontofrist}}$}

\imgw{Bilder/Skonto}{Skontofrist - Zahlungsziel}{}{9cm}

\paragraph{Beispielrechnung}
\begin{enumerate}
	\item Berechnung Skontobetrag
	\item\label{skonto2} Berechnung Jahreszinssatz für Skontobetrag (Dreisatz)\\
	$\text{Jahreszinssatz} = \dfrac{360\cdot\text{Skontosatz}}{\text{Zahlungsziel} - \text{Skontofrist}}$
	\item Berechnung des effektiven Jahreszinssatzes\\
	Zins aus \ref*{skonto2}. bezieht sich auf Rechnungsbetrag\\
	$\Rightarrow$ umrechnen auf um Skonto geminderten Betrag.\\
	$\text{effektiver Jahreszinssatz} = \text{Jahreszinssatz}\cdot\dfrac{100}{100 - \text{Skontosatz}}$
	\item Vergleich mit Bankkredit/\ldots
	\begin{compactitem}
		\item Kosten/Zinsen für Kredit berechnen (Dauer = $\text{Zahlungsziel} - \text{Skontofrist}$)
		\item $\text{Ersparnis} = \text{Skontobetrag} - \text{Kreditkosten}$
	\end{compactitem}
\end{enumerate}

\clearpage
\chapter{Bilanz und Buchungssätze}
\section{Konten}
dienen der Ordnung von Finanzvorfällen (Zahlungen, \ldots)

\paragraph{Kontengruppen} Aktiv, Passiv, Aufwand und Ertrag

\paragraph{Bilanz}\quad\\
\begin{tabular}{c|c}
	Aktiva & Passiva\\\hline
	AV (Anlagevermögen) & EK (Eigenkapital)\\
	UV (Umlaufvermögen) & Schulden\\
	&\\
	&\\\hline
	\ldots & \ldots\\
\end{tabular}

\paragraph{Aktiv Konto}\quad\\
\begin{tabular}{c|c}
	S & H\\\hline
	AB (Anfangsbestand) & \\
	Zugänge (Mehrungen) & Abgänge (Minderungen)\\
	& SB (Schlussbestand) = Summe(Soll) - Summe(Haben)\\
	\circled{1} $\downarrow$ $\sum$ & $\uparrow$ \circled{3}\\
	\circled{2} $\rightarrow$ &\\
	&\\
	$+$ & $-$\\
\end{tabular}

\paragraph{Passiv Konto}\quad\\
\begin{tabular}{c|c}
	S & H\\\hline
	& AB (Anfangsbestand)\\
	Abgänge (Minderungen) & Zugänge (Mehrungen)\\
	SB (Schlussbestand) = Summe(Haben) - Summe(Soll) &\\
	$\uparrow$ \circled{3} & \circled{1} $\downarrow$ $\sum$\\
	& $\leftarrow$ \circled{2}\\
	&\\
	$-$ & $+$\\
\end{tabular}

\paragraph{Einteilung}
\begin{tabbing}
	\hspace{5cm}\=\hspace{2cm}\=\kill
	Aufwandskonto\>Aktiv\>$\text{AB}=0$\\
	Ertragskonto\>Passiv\>$\text{AB}=0$\\
	Eigenkapital\>Passiv\\
	Umsatzsteuer\>Passiv\\
	Vorsteuer\>Aktiv\\
	aRAG\>Aktiv\>(Aufwände) (aktive Rechnungsabgrenzung)\\
	pRAG\>Passiv\>(Erträge) (passive Rechnungsabgrenzung)\\
	Umsatzkonto\>Passiv\\
	GuV\>Passiv\\
	Verbindlichkeiten\>Passiv\\
	Forderungen\>Aktiv\\
\end{tabbing}

\paragraph{Buchungssätze} Per \emph{SOLL} an \emph{HABEN} \emph{xxx \euro}.

Aktivtausch berührt zwei Aktivkonten. (Billanzsumme bleibt bleich)\\
Passivtausch berührt zwei Passivkonten. (Billanzsumme bleibt bleich)\\
Aktiv-Passiv Mehrung berührt ein Aktiv- und ein Passivkonto. (Billanzsumme wird mehr)\\
Aktiv-Passiv Minderung berührt ein Aktiv- und ein Passivkonto. (Billanzsumme wird weniger)

\paragraph{Eigenkapital =} Aktiva - Passiva (aus Bilanz)