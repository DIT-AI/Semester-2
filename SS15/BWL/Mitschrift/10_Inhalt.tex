\chapter{Einführung}
\section{Ökonomisches Prinzip}
\paragraph{Maximum-Prinzip:} Maximierung des Outputs bei gegebenem Input
\paragraph{Minimum-Prinzip:} Minimierung des Inputs bei gegebenem Output
\paragraph{Optimum-Prinzip:} Maximierung der Differenz von Input und Output

\paragraph{Definition Wirtschaften:} Disponieren über knappe Güter.\\
Planvolle zielgerichtete Tätigkeit knappe Güter bestmöglich nutzen.

\paragraph{Maslowsche Bedürfnispyramide}
\begin{compactitem}
	\item Materielle Bedürfnisse
	\item Immaterielle Bedürfnisse
\end{compactitem}

\imgw{Bilder/Kreislauf}{Wirtschaftskreislauf}{}{12cm}
\imgw{Bilder/Einheiten}{Wirtschaftseinheiten}{}{12cm}
\imgw{Bilder/Prozess}{Wertschöpfungsprozess}{}{13cm}

\clearpage
\paragraph{Marktformen}
\imgw{Bilder/Marktformen}{Marktformen}{}{13cm}

\paragraph{Preisbildung am Markt:} abhängig von Wettbewerber, Lieferanten, Kosten und Kunden

\imgw{Bilder/Ziele}{Ziele}{}{13cm}
