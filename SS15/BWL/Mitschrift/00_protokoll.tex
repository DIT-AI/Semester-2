% ################################################################
% #                                                              #
% # Autor: Michael Epping                                        #
% # E-Mail: michael.epping@uni-muenster.de                       #
% # Version: 1.4                                                 #
% # Datum: Juni 2013                                             #
% # Info: Diese Datei sollte nicht verändert werden.             #
% #    Hier werden die Einstellungen festgelegt und              #
% #    Pakete eingebunden. Alles weitere wird über               #
% #    die Dateien verändert, die mit "0X_" beginnen.            #
% # Copyright: CC0 (macht mit diesen Dateien was ihr wollt)      #
% #    https://creativecommons.org/publicdomain/zero/1.0/deed.de #
% #                                                              #
% ################################################################

% Änderungen 1.2 -> 1.3
% * Bei der Verwendung von texlive2012 gibt es Probleme mit myalphadin.
%   Diese Vorlage für Einträge insLiteraturverzeichnis habe ich durch unsrtdin ersetzt.
% * Da ich jetzt mit TeXlipse arbeite, habe ich ein paar Anpassungen vorgenommen.
%   So ist z.B. der Name der bib-Datei fest vorgegeben, damit auch die Autovervollständigung bei BibTeX-Keys funktioniert.
% * Zusätzliche Kommentare sollten das Arbeiten mit dieser Vorlage erleichtern.
% * Die Vorlage enthält sinnvollen Text und nicht nur nutzlose Platzhalter.

% Änderungen 1.3 -> 1.4
% * Das Paket "ifthenx" gibt es unter Ubuntu 12.04 mit texlive 2009-15 nicht. Als alternative habe ich "xifthen" eingetragen.
% * Tabulatoren habe ich durch Leerzeichen ersetzt. Dadurch bleibt das Layout (Einrücken und Position Kommentare) erhalten, 
%   egal mit welchem Editor man die Dateien öffnet.
% * Ich habe einen Copyright-Vermerk hinzugefügt, nämlich dass es im Prizip keines gibt.
% * Neuer Abschnitt: latexmk
% * Neuer Abschnitt: Verbatim
% * Die Bilddatei "titelseite.jpg" wurde entfernt (wegen Copyright), da die Vorlage ab jetzt öffentlich zugänglich sein soll.
% * "README.txt" im Verzeichnis Bilder wurde erstellt.
% * Literaturangabe der Anleitung zur Optik, Wärmelehre und Atomphysik hinzugefügt. 

% ###############
% # Allgemeines #
% ###############

% Zeilen, die mit einem Prozentzeichen beginnen sind Kommentare. 
% Alle verwendeten Funktionen sind mit solchen Kommentaren versehen, so dass man den Zweck der jeweiligen Funktion nachvollziehen kann.

% ######################################
% # Konfigurieren der Dokumentenklasse #
% ######################################

\documentclass[
    a5paper,                                               % Papierformat
    landscape,
    oneside,                                               % Einseitig
    %twoside,                                              % Zweiseitig
    12pt,                                                  % Schriftgröße
    pagesize=auto,                                         % schreibt die Papiergröße korrekt ins Ausgabedokument
    headsepline,                                           % Linie unter der Kopfzeile
    %draft=true                                            % Markiert zu lange und zu kurze Zeilen
]{scrreprt}
% Es gibt die Dokumenttypen scrartcl, srcbook, scrreprt und scrlettr. Diese gehören zum KOME-Skript und sollten für deutsche Texte benutzt werden.
% Für englische Texte wählt man entsprechend article, book, report und letter.
% Es ist  nicht unbedingt zu empfehlen, bei einem bestehendem Dokument, die documentclass zu ändern.

% ####################
% # Pakete einbinden #
% ####################

% ####################
% # Pakete einbinden #
% ####################

% Pakete erweitern LaTeX um zusätzliche Funktionen. Dies ist eine Satz nützlicher Pakete.
% Weitere sollten in der Datei"`01_EigenePakete.tex"' hinzugefügt werden.
\usepackage[utf8x]{inputenc}                                                  % Legt die Zeichenkodierung fest, z.B UTF8
\usepackage[T1]{fontenc}                                                      % Verwendung der Zeichentabelle T1, für deutschsprachige Dokumente sinnvoll
\usepackage[ngerman,english]{babel}                                           % Silbentrennung nach neuer deutscher und englischer Rechtschreibung
\usepackage{amsmath}                                                          % Mathepaket
\usepackage{amsfonts}
\usepackage{amssymb}                                                          % Mathepaket
\usepackage{xifthen}                                                          % Wird benötigt um \ifthenelse zu benutzen
%\usepackage{graphicx}                                                         % Zum flexiblen Einbinden von Grafiken, pdftex ist optional
\usepackage[pdftex]{graphicx}                                                 % Zum flexiblen Einbinden von Grafiken, pdftex ist optional
\usepackage{units}                                                            % Ermöglicht die Nutzung von \unit[Zahl]{Einheit}
\usepackage{setspace}                                                         % Einfaches wechseln zwischen unterschiedlichen Zeilenabständen
%\usepackage{hyperref}                                                         % Verlinkt Textstellen im PDF Dokument
\usepackage[pdfpagelabels]{hyperref}                                          % Verlinkt Textstellen im PDF Dokument
\usepackage[font=small,labelfont=bf,labelsep=endash,format=plain]{caption}    % Darstellung für Caption s.u.
\usepackage{subfig}                                                           % Bilder nebeneinander
\usepackage{wrapfig}                                                          % Fließtext um Figure-Umgebung
\usepackage{cite}                                                             % Zusatzfunktionen zum zitieren
\usepackage{scrpage2}                                                         % Wird für Kopf- und Fußzeile benötigt
\usepackage{array,dcolumn}                                                    % Beide Pakete werden für die Ausrichtung der Tabellenspalten benötigt


% ############################
% # weitere Pakete einbinden #
% ############################

% \usepackage{showframe}
% Die Folgenden Pakete sind schon eingebunden (siehe 00_Protokoll.tex):
% \usepackage[utf8x]{inputenc}                             % Legt die Zeichenkodierung fest, z.B UTF8
% \usepackage[T1]{fontenc}                                 % Verwendung der Zeichentabelle T1, für deutschsprachige Dokumente sinnvoll
% \usepackage[ngerman,english]{babel}                      % Silbentrennung nach neuer deutscher und englischer Rechtschreibung
% \usepackage{amsmath}                                     % Mathepaket
% \usepackage{amssymb}                                     % Mathepaket
% \usepackage{ifthenx}                                     % Wird benötigt um \ifthenelse zu benutzen
% \usepackage[pdftex]{graphicx}                            % Zum flexiblen Einbinden von Grafiken, pdftex ist optional
% \usepackage{rotating}                            % Zum drehen von Objekten, pdftex ist optional
  \usepackage[pdftex]{rotating}                            % Zum drehen von Objekten, pdftex ist optional
% \usepackage{units}                                       % Ermöglicht die Nutzung von \unit[Zahl]{Einheit}
% \usepackage{setspace}                                    % Einfaches wechseln zwischen unterschiedlichen Zeilenabständen
% \usepackage[pdfpagelabels]{hyperref}                     % Verlinkt Textstellen im PDF Dokument
% \usepackage[font=small,labelfont=bf,labelsep=endash,format=plain]{caption}
%                                                          % Darstellung für Caption s.u.
% \usepackage{subfig}                                      % Bilder nebeneinander
% \usepackage{wrapfig}                                     % Fließtext um Figure-Umgebung
% \usepackage{cite}                                        % Zusatzfunktionen zum zitieren
% \usepackage{scrpage2}                                    % Wird für Kopf- und Fußzeile benötigt
% \usepackage{array,dcolumn}                               % Beide Pakete werden für die Ausrichtung der Tabellenspalten benötigt
  \usepackage{enumerate}                                   % Um andere Aufzählungsvarianten zu erzeugen http://ctan.org/pkg/enumerate
  \usepackage{xcolor}
  \usepackage{ulem}
% \usepackage{mathtools}
  \usepackage{longtable}
  \usepackage{tabularx}                                    % http://ctan.org/pkg/tabularx
  \usepackage{booktabs}                                    % http://ctan.org/pkg/booktabs
% \usepackage{a4wide}
  \usepackage{geometry}
  \usepackage{amsthm}
% \usepackage{pstricks-add}
% \usepackage{pstricks}
  \usepackage{pgf,tikz}
  \usetikzlibrary{arrows}
  \usepackage{chngcntr}
  \usepackage{trfsigns}
  \usepackage{multicol}
  \usepackage{eurosym}
  \usepackage{pdfpages}
  \usepackage[nomessages]{fp}% http://ctan.org/pkg/fp
  \usepackage{calc}
  \usepackage{listings}
  \usepackage{paralist}
\input{kvmacros}

% Spezialpakete
\usepackage{fp}
\usepackage{tikz}
\usepackage{xcolor}
% TikZ-Bibliotheken
\usetikzlibrary{arrows}
\usetikzlibrary{shapes}
\usetikzlibrary{decorations.pathmorphing}
\usetikzlibrary{decorations.pathreplacing}
\usetikzlibrary{decorations.shapes}
\usetikzlibrary{decorations.text}
\usetikzlibrary{calc}
\usetikzlibrary{snakes}
\usetikzlibrary{positioning}



% ############################
% # Eigene Befehle einbinden #
% ############################

\allowdisplaybreaks

\newcolumntype{E}{>{\begin{math}\displaystyle}c<{\end{math}}}%e wie Equation
\newcolumntype{K}{m{0.45\textwidth}}% K wie kommentar

\newcolumntype{L}[1]{>{\raggedright\let\newline\\\arraybackslash\hspace{0pt}}m{#1}}
\newcolumntype{C}[1]{>{\centering\let\newline\\\arraybackslash\hspace{0pt}}m{#1}}
\newcolumntype{R}[1]{>{\raggedleft\let\newline\\\arraybackslash\hspace{0pt}}m{#1}}

\geometry{a5paper, landscape, left=25mm, right=20mm, top=22mm, bottom=15mm} 

\counterwithin{chapter}{part}
\renewcommand{\thechapter}{\arabic{chapter}}

\setcounter{secnumdepth}{4} % seting level of numbering (default for "report" is 3). With ''-1'' you have non number also for chapters
%\setcounter{tocdepth}{5} % if you want all the levels in your table of contents

\setlength{\parindent}{0pt}
\setlength{\parskip}{12pt}

% ############################
% # Eigene Befehle einbinden #
% ############################

% Eigene Befehle eignen sich gut um Abkürzungen für lange Befehle zu erstellen. Die Syntax ist folgende:
% \newcommand{neuer Befahl}{ein langer Befehl}
% Das folgende Beispiel fügt ein Bild mit bestimmten vorgegebenen Optionen ein:
\newcommand{\cImage}[1]{
	\begin{figure}[h!]
		\centering
		\includegraphics[width=0.50\textwidth]{#1}
	\end{figure}
}

% Images
\newcommand{\imgplaceholder}{
	\begin{center}
		\begin{figure}[h!]
			\centering \fcolorbox{orange}{orange}{\white{Hier kommt noch ein Bild}}
			\caption{}
			\label{}
		\end{figure}
	\end{center}
}
\newcommand{\img}[3]{
	\begin{center}
		\begin{figure}[h!]
			\centering \includegraphics[width=\linewidth]{#1}
			\caption{#2}
			\label{#3}
		\end{figure}
	\end{center}
}
\newcommand{\imgh}[4]{
	\begin{center}
		\begin{figure}[h!]
			\centering \includegraphics[height=#4]{#1}
			\caption{#2}
			\label{#3}
		\end{figure}
	\end{center}
}
\newcommand{\imgw}[4]{
	\begin{center}
		\begin{figure}[h!]
			\centering \includegraphics[width=#4]{#1}
			\caption{#2}
			\label{#3}
		\end{figure}
	\end{center}
}
\newcommand{\imgwh}[5]{
	\begin{center}
		\begin{figure}[h!]
			\centering \includegraphics[width=#4,height=#5]{#1}
			\caption{#2}
			\label{#3}
		\end{figure}
	\end{center}
}

% #1 ist dabei ein Parameter, den man \cImage übergeben muss. In 10_Titelseite.tex wird dieser Befehl verwendet. Der Parameter ist dort Bilder/titelseite.jpg.
% Benötigt man keine Parameter, dann lässt man [1] weg. Werden zusätzliche Parameter benötigt, dann kann man die Zahl auf maximal 9 erhöhen.

% Disjunkte Vereinigung
\makeatletter
\def\moverlay{\mathpalette\mov@rlay}
\def\mov@rlay#1#2{\leavevmode\vtop{%
		\baselineskip\z@skip \lineskiplimit-\maxdimen
		\ialign{\hfil$\m@th#1##$\hfil\cr#2\crcr}}}
\newcommand{\charfusion}[3][\mathord]{
	#1{\ifx#1\mathop\vphantom{#2}\fi
		\mathpalette\mov@rlay{#2\cr#3}
	}
	\ifx#1\mathop\expandafter\displaylimits\fi}
\makeatother

\newcommand{\cupdot}{\charfusion[\mathbin]{\cup}{\cdot}}
\newcommand{\bigcupdot}{\charfusion[\mathop]{\bigcup}{\cdot}}
\newcommand{\bigX}{\charfusion[\mathop]{{\LARGE\text{$\times$}}}{\ }}

% circled text
\newcommand{\circled}[1]{\tikz[baseline=(char.base)]{
		\node[shape=circle,draw,inner sep=2pt] (char) {#1};}}
% go high
\newcommand{\gohigh}{\tikz{
		\draw (0,0) -- (0.5em,0) -- (0.5em,1em) -- (1em,1em);}}
% go low
\newcommand{\golow}{\tikz{
			\draw (0,1em) -- (0.5em,1em) -- (0.5em,0) -- (1em,0);}}

\newcommand{\tx}[1]{\text{#1}}

% Text Color
\definecolor{mygreen}{rgb}{0.1, 0.65, 0.1}
\definecolor{amethyst}{rgb}{0.6, 0.4, 0.8}
\newcommand{\orange}[1]{\textcolor{orange}{#1}}
\newcommand{\red}[1]{\textcolor{red}{#1}}
\newcommand{\blue}[1]{\textcolor{blue}{#1}}
\newcommand{\green}[1]{\textcolor{mygreen}{#1}}
\newcommand{\violett}[1]{\textcolor{amethyst}{#1}}
\newcommand{\gray}[1]{\textcolor{gray}{#1}}
\newcommand{\brown}[1]{\textcolor{brown}{#1}}
\newcommand{\black}[1]{\textcolor{black}{#1}}
\newcommand{\white}[1]{\textcolor{white!0}{#1}}

% Underline Color
\newcommand{\orangeul}[1]{\colorlet{saved}{.}\orange{\ul{\color{saved}#1}}}
\newcommand{\redul}[1]{\colorlet{saved}{.}\red{\ul{\color{saved}#1}}}
\newcommand{\blueul}[1]{\colorlet{saved}{.}\blue{\ul{\color{saved}#1}}}
\newcommand{\greenul}[1]{\colorlet{saved}{.}\green{\ul{\color{saved}#1}}}
\newcommand{\violettul}[1]{\colorlet{saved}{.}\violett{\ul{\color{saved}#1}}}
\newcommand{\grayul}[1]{\colorlet{saved}{.}\gray{\ul{\color{saved}#1}}}
\newcommand{\brownul}[1]{\colorlet{saved}{.}\brown{\ul{\color{saved}#1}}}
\newcommand{\blackul}[1]{\colorlet{saved}{.}\black{\ul{\color{saved}#1}}}
\newcommand{\whiteul}[1]{\colorlet{saved}{.}\white{\ul{\color{saved}#1}}}

% Underbrace Color
\newcommand{\redunderbrace}[2]{\colorlet{saved}{.}\red{\underbrace{\color{saved}#1}_{#2}}}
\newcommand{\blueunderbrace}[2]{\colorlet{saved}{.}\blue{\underbrace{\color{saved}#1}_{#2}}}
\newcommand{\greenunderbrace}[2]{\colorlet{saved}{.}\green{\underbrace{\color{saved}#1}_{#2}}}
\newcommand{\grayunderbrace}[2]{\colorlet{saved}{.}\gray{\underbrace{\color{saved}#1}_{#2}}}

% Overbrace Color
\newcommand{\redoverbrace}[2]{\colorlet{saved}{.}\red{\overbrace{\color{saved}#1}_{#2}}}
\newcommand{\blueoverbrace}[2]{\colorlet{saved}{.}\blue{\overbrace{\color{saved}#1}_{#2}}}
\newcommand{\greenoverbrace}[2]{\colorlet{saved}{.}\green{\overbrace{\color{saved}#1}_{#2}}}
\newcommand{\grayoverbrace}[2]{\colorlet{saved}{.}\gray{\overbrace{\color{saved}#1}_{#2}}}

% Under
\newcommand{\under}[1]{\colorlet{saved}{.}\white{\underbrace{\black{#1}}}}
\newcommand{\redunder}[2]{\colorlet{saved}{.}\white{\underbrace{\black{#1}}_{\red{#2}}}}
\newcommand{\blueunder}[2]{\colorlet{saved}{.}\white{\underbrace{\black{#1}}_{\blue{#2}}}}
\newcommand{\greenunder}[2]{\colorlet{saved}{.}\white{\underbrace{\black{#1}}_{\green{#2}}}}
\newcommand{\grayunder}[2]{\colorlet{saved}{.}\white{\underbrace{\black{#1}}_{\gray{#2}}}}

% Over
\renewcommand{\over}[1]{\colorlet{saved}{.}\white{\overbrace{\black{#1}}}}
\newcommand{\redover}[2]{\colorlet{saved}{.}\white{\overbrace{\black{#1}}_{\red{#2}}}}
\newcommand{\blueover}[2]{\colorlet{saved}{.}\white{\overbrace{\black{#1}}_{\blue{#2}}}}
\newcommand{\greenover}[2]{\colorlet{saved}{.}\white{\overbrace{\black{#1}}_{\green{#2}}}}
\newcommand{\grayover}[2]{\colorlet{saved}{.}\white{\overbrace{\black{#1}}_{\gray{#2}}}}

% cBoxes
\newcommand{\cbox}[1]{{\vspace{-0.4cm}\begin{center}#1\end{center}\vspace{-0.4cm}}}
\newcommand{\graycbox}[1]{{\vspace{-0.4cm}\begin{center}\vspace{2pt}\fcolorbox{black!80}{black!10}{#1}\vspace{2pt}\end{center}\vspace{-0.4cm}}}
\newcommand{\redcbox}[1]{{\vspace{-0.4cm}\begin{center}\vspace{2pt}\fcolorbox{red}{red!15}{#1}\vspace{2pt}\end{center}\vspace{-0.4cm}}}
\newcommand{\greencbox}[1]{{\vspace{-0.4cm}\begin{center}\vspace{2pt}\fcolorbox{green}{green!15}{#1}\vspace{2pt}\end{center}\vspace{-0.4cm}}}
\newcommand{\bluecbox}[1]{{\vspace{-0.4cm}\begin{center}\vspace{2pt}\fcolorbox{blue}{blue!15}{#1}\vspace{2pt}\end{center}\vspace{-0.4cm}}}

% Boxes
\newcommand{\graybox}[1]{{\vspace{2pt}\fcolorbox{black!80}{black!10}{#1}\vspace{2pt}}}
\newcommand{\redbox}[1]{{\vspace{2pt}\fcolorbox{red}{red!15}{#1}\vspace{2pt}}}
\newcommand{\greenbox}[1]{{\vspace{2pt}\fcolorbox{green}{green!15}{#1}\vspace{2pt}}}
\newcommand{\bluebox}[1]{{\vspace{2pt}\fcolorbox{blue}{blue!15}{#1}\vspace{2pt}}}
\newcommand{\Graybox}[1]{\fcolorbox{black!80}{black!10}{\parbox{\dimexpr\linewidth-2\fboxsep-2\fboxrule\relax}{#1}}}
\newcommand{\Redbox}[1]{\fcolorbox{red}{red!15}{\parbox{\dimexpr\linewidth-2\fboxsep-2\fboxrule\relax}{#1}}}
\newcommand{\Greenbox}[1]{\fcolorbox{green}{green!15}{\parbox{\dimexpr\linewidth-2\fboxsep-2\fboxrule\relax}{#1}}}
\newcommand{\Bluebox}[1]{\fcolorbox{blue}{blue!15}{\parbox{\dimexpr\linewidth-2\fboxsep-2\fboxrule\relax}{#1}}}
% Frames
\newcommand{\grayframe}[1]{{\color{black!30}\fbox{\color{black}#1}}}
\newcommand{\redframe}[1]{{\color{red}\fbox{\color{black}#1}}}
\newcommand{\greenframe}[1]{{\color{green}\fbox{\color{black}#1}}}
\newcommand{\blueframe}[1]{{\color{blue}\fbox{\color{black}#1}}}
\newcommand{\Grayframe}[1]{{\color{black!30}\fbox{\color{black}\parbox{\dimexpr\linewidth-2\fboxsep-2\fboxrule\relax}{#1}}}}
\newcommand{\Redframe}[1]{{\color{red}\fbox{\color{black}\parbox{\dimexpr\linewidth-2\fboxsep-2\fboxrule\relax}{#1}}}}
\newcommand{\Greenframe}[1]{{\color{green}\fbox{\color{black}\parbox{\dimexpr\linewidth-2\fboxsep-2\fboxrule\relax}{#1}}}}
\newcommand{\Blueframe}[1]{{\color{blue}\fbox{\color{black}\parbox{\dimexpr\linewidth-2\fboxsep-2\fboxrule\relax}{#1}}}}

% Bsp
%\grayframe{grayframe}
%\redframe{redframe}
%\greenframe{greenframe}
%\Grayframe{Grayframe}
%\Redframe{Redframe}
%\Greenframe{Greenframe}
%\graybox{graybox}
%\redbox{redbox}
%\greenbox{greenbox}
%\Graybox{Graybox}
%\Redbox{Redbox}
%\Greenbox{Greenbox}



% Usefull
\renewcommand{\emph}[1]{\textit{#1}}

\newtheoremstyle{satzstyle}{15pt}{15pt}{}{-42pt}{\bf}{:}{1em}{}
\theoremstyle{satzstyle}
\newtheorem*{satz}{Satz}
\newtheoremstyle{korollarstyle}{15pt}{15pt}{}{-66pt}{\bf}{:}{1em}{}
\theoremstyle{korollarstyle}
\newtheorem*{korollar}{Korollar}
\newtheoremstyle{theoremstyle}{15pt}{15pt}{}{-67pt}{\bf}{:}{1em}{}
\theoremstyle{theoremstyle}
\newtheorem*{theorem}{Theorem}
\newtheoremstyle{lemmastyle}{15pt}{15pt}{}{-59pt}{\bf}{:}{1em}{}
\theoremstyle{lemmastyle}
\newtheorem*{lemma}{Lemma}
\newcommand{\pp}{\marginpar[\red{\bf (P)}]{\red{\bf (P)}}}
\newcommand{\PP}{\red{\bf Prüfungsrelevant (P)\quad}\marginpar[\red{\bf (P)}]{\red{\bf (P)}}}
\newcommand{\Def}{\paragraph{Definition:}}
\newcommand{\Bem}{\paragraph{Bemerkung:}}
\newcommand{\Bsp}{\paragraph{Beispiel:}}
\newcommand{\Bsps}{\paragraph{Beispiele:}}
\newcommand{\Beachte}{\paragraph{Beachte:}}
\newcommand{\Beweis}{\paragraph{Beweis:}}
\newcommand{\Formel}{\paragraph{Formel:}}
\newcommand{\Problem}{\paragraph{Problem:}}
\newcommand{\Beh}{\paragraph{Behauptung:}}
\newcommand{\Prosa}{\paragraph{Prosa:}}
\newcommand{\Frage}{\paragraph{Frage:}}
\newcommand{\Idee}{\paragraph{Idee:}}
\newcommand{\Allg}{\paragraph{Allgemein:}}
\newcommand{\Trick}{\paragraph{Trick:}}
\newcommand{\Tipp}{\paragraph{Tipp:}}
\newcommand{\Merke}{\paragraph{Merke:}}
\newcommand{\Ziel}{\paragraph{Ziel:}}
\newcommand{\Praxis}{\paragraph{Praxis:}}
%\newcommand{\Satz}[1]{\paragraph{Satz:}\graybox{#1\hfill}}
%\newcommand{\Satz}[1]{\paragraph{Satz:}
%	
%	\Graybox{#1}}
\newcommand{\Satz}[1]{\begin{satz}\begin{minipage}{\linewidth}\Graybox{#1}\end{minipage}\end{satz}}
%\newcommand{\SATZ}[1]{\paragraph{Satz:}
%	
%	\Graybox{#1}}
\newcommand{\Korollar}[1]{\begin{korollar}\Graybox{#1}\end{korollar}}
%\newcommand{\Korollar}[1]{\paragraph{Korollar:}
%	
%	\Graybox{#1}}
\newcommand{\Theorem}[1]{\begin{theorem}\Graybox{#1}\end{theorem}}
%\newcommand{\Theorem}[1]{\paragraph{Theorem:}
%	
%	\Graybox{#1}}
\newcommand{\Lemma}[1]{\begin{lemma}\Graybox{#1}\end{lemma}}
%\newcommand{\Lemma}[1]{\paragraph{Lemma:}
%	
%	\Graybox{#1}}

% Math
\newcommand{\N}{\mathbb{N}}
\newcommand{\No}{\mathbb{N}_0}
\newcommand{\Z}{\mathbb{Z}}
\newcommand{\Zp}{\mathbb{Z}^+}
\newcommand{\Zop}{\mathbb{Z}_0^+}
\newcommand{\Zm}{\mathbb{Z}^-}
\newcommand{\Zom}{\mathbb{Z}_0^-}
\newcommand{\Q}{\mathbb{Q}}
\newcommand{\Qp}{\mathbb{Q}^+}
\newcommand{\Qop}{\mathbb{Q}_0^+}
\newcommand{\Qm}{\mathbb{Q}^-}
\newcommand{\Qom}{\mathbb{Q}_0^-}
\newcommand{\R}{\mathbb{R}}
\newcommand{\Rp}{\mathbb{R}^+}
\newcommand{\Rop}{\mathbb{R}_0^+}
\newcommand{\Rm}{\mathbb{R}^-}
\newcommand{\Rom}{\mathbb{R}_0^-}
\newcommand{\C}{\mathbb{C}}
\newcommand{\F}{\mathbb{F}}
\newcommand{\G}{\mathbb{G}}
\newcommand{\D}{\mathbb{D}}
\renewcommand{\L}{\mathbb{L}}

% Grad
\newcommand{\degree}{{^\circ}}

% Differenzial
\newcommand{\diff}[1]{\mathrm{d}#1}
\newcommand{\df}[2]{\dfrac{\diff{#1}}{\diff{#2}}}
\newcommand{\ddf}[2]{\dfrac{\diff{#1}}{\diff{#2}^2}}
% Integral d korrekt formatieren
\newcommand{\intd}[1]{\ \diff{#1}}
% ^-1
\newcommand{\inv}[1]{{#1}^{-1}}
% Entspricht
\newcommand{\entspricht}{\stackrel{\wedge}{=}}
% Lines
\newcommand{\oline}[1]{\overline{#1}}
\newcommand{\ol}[1]{\overline{#1}}
%\newcommand{\uline}[1]{\underline{#1}}
\newcommand{\ul}[1]{\underline{#1}}
% Displaystyle
\newcommand{\ds}{\displaystyle}
% i
\renewcommand{\i}{\imath}
\newcommand{\ii}{\imath^2}
% Arg()
\newcommand{\Arg}{\mathbin\text{Arg}}
% Ln()
\newcommand{\Ln}{\mathbin\text{Ln}}
% GF()
\newcommand{\GF}{\mathbin\text{GF}}
% ggT(), kgV()
\newcommand{\ggT}{\mathbin\text{ggT}}
\newcommand{\kgV}{\mathbin\text{kgV}}
% sgn()
\newcommand{\sgn}{\mathbin\text{sgn}}
% grad()
\newcommand{\grad}{\mathbin\text{grad}}
%
\newcommand{\arccot}{\mathbin\text{arccot}}
\newcommand{\arsinh}{\mathbin\text{arsinh}}
\newcommand{\arcosh}{\mathbin\text{arcosh}}
\newcommand{\artanh}{\mathbin\text{artanh}}
\newcommand{\arcoth}{\mathbin\text{arcoth}}
% impliziert, äquivalent
\newcommand{\imp}{\rightarrow}
\newcommand{\eq}{\leftrightarrow}

%
\newcommand{\lxor}{\oplus}

% Vectoren und Matrizen
\newcommand{\vv}[2]{$\begin{pmatrix}#1\\#2\end{pmatrix}$}
\newcommand{\vvv}[3]{$\begin{pmatrix}#1\\#2\\#3\end{pmatrix}$}
\newcommand{\vvvv}[4]{$\begin{pmatrix}#1\\#2\\#3\\#4\end{pmatrix}$}

\newcommand{\vvs}[2]{$\begin{pmatrix}#1&#2\end{pmatrix}$}
\newcommand{\vvvs}[3]{$\begin{pmatrix}#1&#2&#3\end{pmatrix}$}
\newcommand{\vvvvs}[4]{$\begin{pmatrix}#1&#2&#3&#4\end{pmatrix}$}

\newcommand{\mm}[4]{$\begin{pmatrix}#1&#2\\#3&#4\end{pmatrix}$}
\newcommand{\mmm}[9]{$\begin{pmatrix}#1&#2&#3\\#4&#5&#6\\#7&#8&#9\end{pmatrix}$}

\newcommand{\mmmm}[8]{
	\def\Argi{{#1}}%
	\def\Argii{{#2}}%
	\def\Argiii{#3}%
	\def\Argiv{#4}%
	\def\Argv{#5}%
	\def\Argvi{#6}%
	\def\Argvii{#7}%
	\def\Argviii{#8}%
	\mmmmContinue
}

\newcommand{\mmmmContinue}[8]{$\begin{pmatrix}\Argi&\Argii&\Argiii&\Argiv\\\Argv&\Argvi&\Argvii&\Argviii\\#1&#2&#3&#4\\#5&#6&#7&#8\end{pmatrix}$}



% #########################
% # Variablen importieren #
% #########################

% Der Befehl \newcommand kann auch benutzt werden um Variablen zu definieren:

% Semester:
\newcommand{\varSemester}{2. Semester}
% Fach:
\newcommand{\varFach}{Algorithmen und Datenstrukturen}
\newcommand{\varFachShort}{AuD}
% Teilgebiet:
%\newcommand{\varTeilGebiet}{Teilgebiet}
% Thema:
%\newcommand{\varThema}{Thema}
% Datum:
%\newcommand{\varDate}{\today}
% Dozent:
\newcommand{\varDozent}{Prof. Dr. rer. nat. Peter Jüttner}
% Autor:
\newcommand{\varAutor}{Christoph Stephan}
% E-Mail-Adresse des Dozenten:
\newcommand{\varEmailDozent}{peter.juettner@th-deg.de}
% E-Mail-Adresse des Autors:
\newcommand{\varEmailAutor}{chm.stephan@outlook.com}
% E-Mail-Adresse anzeigen (true/false):
\newcommand{\varZeigeEmail}{true}
% Anhang anzeigen (true/false):
\newcommand{\varZeigeAnhang}{false}
% Literaturverzeichnis anzeigen (true/false):
\newcommand{\varZeigeLiteraturverzeichnis}{false}
% Stil der Einträge im Literaturverzeichnis
\newcommand{\varLiteraturLayout}{unsrtdin}

\newboolean{showLiteratur}
\newboolean{showAnhang}

% #########################
% # Beginn des Dokumentes #
% #########################

\begin{document}
\selectlanguage{ngerman}                                   % Schreibsprache Deutsch
\onehalfspacing                                            % 1 1/2 facher Zeilenabstand
\addtokomafont{sectioning}{\rmfamily}                      % Schriftsatz
\numberwithin{equation}{section}                           % Nummerierung der Formeln entsprechend der Section (z.B. 1.1)
\addtokomafont{caption}{\small\linespread{1}\selectfont}   % ändert Schriftgröße und Zeilenabstand bei captions

% Römische Ziffern als Seitenzahlen für Titelseite bis einschließlich dem Inhaltsverzeichnis
\setcounter{page}{1}
\pagenumbering{roman}

% #######################################
% # Kopf- und Fußzeile konfigurieren    #
% #######################################

%\ihead{\textit{\varFachShort\ - \varSemester}}             % Innenseite der Kopfzeile
%\chead{}                                                   % Mitte der Kopfzeile
%\ohead{\textit{\varDozent}}                                % Außenseite der Kopfzeile
%\ifoot{}                                                   % Innnenseite der Fußzeile
%\cfoot{- \textit{\pagemark} -}                             % Mitte der Fußzeile
%\ofoot{}                                                   % Aussenseite der Fußzeile
\ihead{\textit{\thechapter\ \leftmark}}                    % Innenseite der Kopfzeile
\chead{}                                                   % Mitte der Kopfzeile
\ohead{\textit{\small\thesection\ \rightmark}}             % Außenseite der Kopfzeile
\ifoot{}                                                   % Innnenseite der Fußzeile
\cfoot{- \textit{\pagemark} -}                             % Mitte der Fußzeile
\ofoot{}                                                   % Aussenseite der Fußzeile

\setlength{\footskip}{20pt}

\renewcommand{\chaptermark}[1]{%
	\markleft{#1}}
\renewcommand{\sectionmark}[1]{%
	\markright{#1}}

% ###################################
% # Ausrichtung der Tabellenspalten #
% ###################################

\newcolumntype{,}[1]{D{,}{,}{#1}}                          % , in Tabellen untereinander stellen
\newcolumntype{p}{D{p}{\pm}{-1}}                           % +- in Tabellen untereinander stellen

% ########################
% # Titelseite einbinden #
% ########################

\begin{titlepage}
	\vspace*{0cm}
	\begin{center}
		\Huge
		\textbf{\varFach} \\
		\large{\varSemester} \\
		
		\vspace{0.5cm}
		
		%Vorlesung vom \varDate \\
		
		%\huge{\varTeilGebiet} \\
		
		%\LARGE{\varThema} \\
		
		\vspace{0cm}
		
		\IfFileExists{Bilder/titelseite}{
			\cImage{Bilder/titelseite}
		} % Nach \IfFileExists muss eine Leerzeile eingefügt werden
		
		\vspace{0,5cm}
		
		\newboolean{showEmail}
		\setboolean{showEmail}{\varZeigeEmail}
		
		\small{Dozent} \\
		\Large{\textit{\varDozent}} \\  
		\normalsize
		\ifthenelse{\boolean{showEmail}}{\textit{\varEmailDozent}\\}{}
		
		\vspace{0.5cm}
		
		\small{Mitschrift} \\
		\Large{\textit{\varAutor}} \\
		\normalsize
		\ifthenelse{\boolean{showEmail}}{\textit{\varEmailAutor}\\}{}  
	\end{center}
\end{titlepage}


% ################################
% # Inhaltsverzeichnis einbinden #
% ################################

\tableofcontents
\clearpage

% Zurücksetzen der Seitenzahlen auf arabische Ziffern
\setcounter{page}{1}
\pagenumbering{arabic}

\pagestyle{scrheadings}                                    % Ab hier mit Kopf- und Fußzeile

% ###################################
% # Den Inhalt der Arbeit einbinden #
% ###################################

%\renewcommand{\thesection}{\Roman{section}} 
\renewcommand{\labelitemi}{-}
\chapter{Einführung}
\section{Ökonomisches Prinzip}
\paragraph{Maximum-Prinzip:} Maximierung des Outputs bei gegebenem Input
\paragraph{Minimum-Prinzip:} Minimierung des Inputs bei gegebenem Output
\paragraph{Optimum-Prinzip:} Maximierung der Differenz von Input und Output

\paragraph{Definition Wirtschaften:} Disponieren über knappe Güter.\\
Planvolle zielgerichtete Tätigkeit knappe Güter bestmöglich nutzen.

\paragraph{Maslowsche Bedürfnispyramide}
\begin{compactitem}
	\item Materielle Bedürfnisse
	\item Immaterielle Bedürfnisse
\end{compactitem}

\imgw{Bilder/Kreislauf}{Wirtschaftskreislauf}{}{12cm}
\imgw{Bilder/Einheiten}{Wirtschaftseinheiten}{}{12cm}
\imgw{Bilder/Prozess}{Wertschöpfungsprozess}{}{13cm}

\clearpage
\paragraph{Marktformen}
\imgw{Bilder/Marktformen}{Marktformen}{}{13cm}

\paragraph{Preisbildung am Markt:} abhängig von Wettbewerber, Lieferanten, Kosten und Kunden

\imgw{Bilder/Ziele}{Ziele}{}{13cm}

\clearpage
\chapter{Investitionsrechnung}
\paragraph{Investitionsbegriff:} Investition, Herstellung und Erwerb von Sachgütern

\paragraph{Finanzierungsplan}\quad\\
\ul{Mittelverwendung/Aktiva:} wohin (Anlagevermögen, Umlaufvermögen)
\ul{Mittelherkunft/Passiva:} woher (Eigenkapital, Fremdkapital\red{?}, Rückstellungen, Verbindlichkeiten)

\paragraph{Investitionsarten:} Richtung wie der Geldbetrag angelegt wird.

\begin{compactitem}
	\item Sachinvestitionen (z.B. Grundstücke, Gebäude, Maschinen)
	\item Finanzinvestitionen (z.B. Aktien, Beteiligungen, Pfandbriefe)
	\item Immaterielle Investitionen (z.B. Forschung \& Entwicklung, Aus- \& Weiterbildung, Lizenzen)
\end{compactitem}

\clearpage
\section{Bewertung von Investitionsvorhaben}
\begin{tabular}{l|l}
	{\bf Statistische Verfahren} & {\bf Dynamische Verfahren}\\\hline
	Rentabilitätsrechnung & Barwertverfahren\\
	Kostenvergleichsrechnung & Annuitätenmethode\\
	Gewinnvergleichsmethode & Interner Zins\\
	Amortationsmethode & \\
\end{tabular}

\clearpage
\subsection{Aufzinsen und Abzinsen}
Ausgangspunkt der Investitionsplanung ist die Annahme, dass ein Geldbetrag, der {\flqq heute\frqq} fällig wird, anders zu beurteilen ist als ein gleicher Geldbetrag, der {\flqq morgen\frqq} fällig wird.

Daraus folgt -- Grundlage der Investitionstheorie ist das Auf- und Abzinsen von Zahlungsströmen, die zu verschiedenen Zeitpunkten anfallen.

\paragraph{Aufzinsen}
\redbox{$K_n=K_0\cdot(1+p)^n$}

\paragraph{Abzinsen}
\redbox{$K_0=\dfrac{K_n}{(1+p)^n}$}

\paragraph{Durchschnittszinsen}
\redbox{$p=\sqrt[n]{\dfrac{K_n}{K_0}}-1$}

\paragraph{Investitionszeit} um aus Anfangsbetrag Endbetrag zu machen
\redbox{$n=\dfrac{\log{\left(\dfrac{K_n}{K_0}\right)}}{\log{(1+p)}}$}

\clearpage
\subsubsection{Zusammenfassung}
\begin{compactitem}
	\item Zahlungen werden zeitabhängig bewertet.
	\item Vergleich von verschiedenen Zahlungen immer als hätte man das Kapital am Kapitalmarkt angelegt.
	\item Wesentliche Formel: Aufzinsformel $K_n=K_0\cdot(1+p)^n$
\end{compactitem}

\subsection{Kapitalwertmethode}
Die Kapitalwertmethode setzt voraus, dass der Investor weiß, welchen {\flqq Zinsgewinn\frqq} er aus einem Investitionsobjekt mindestens erwirtschaften will. Dieses vom Marktzins und Risikogesichtspunkten abhängige Mindestverzinsung nennt man Kalkulationszinssatz (p).

Die Kapitalwertmethode prüft, ob in einem Investitionsobjekt mindestens der Kalkulationszinssatz steckt und es sich damit lohnt.

\subsection{Statistische Methoden}
Statistische Methoden nutzen nicht den Zeitwert des Geldes.

\subsubsection{Rentabilitätsrechnung}
Kennzahl die den Projektgewinn in Verhältnis zum Kapitaleinsatz setzt. Gewinn und Kapital als durchschnittlich pro Periode.

Verhältnis der beiden Größen is die Rentabilität des Projekts (durchschnittliche Verzinsung des Projekts).

\greenbox{Liquidationserlös = Schrottwert}

\greenbox{durchschnittlich gebundenes Kapital = Kapitaleinsatz}

\greenbox{kalkulatorische Zinsen = Zinskosten}

\greenbox{UV-Erhöhung = Erhöhung des Umlaufvermögens}

\paragraph{Berechnungen:}\quad\\
\redbox{$\text{Kapitaleinsatz} = \dfrac{\text{Anschaffungswert} + \text{Liquidationserlös} + (\text{UV-Erhöhung})}{2}$} (Durchschnitt aus Anschaffungswert und Schrottwert)\\
\redbox{$\text{Abschreibungen} = \dfrac{\text{Anschaffungswert} - \text{Liquidationserlös} - (\text{UV-Erhöhung})}{\text{Nutzungsdauer}}$}\\
\redbox{$\text{kalkulatorische Zinsen} = \text{Zinsfuß}\cdot\text{Kapitaleinsatz}$}\\

\imgw{Bilder/Investitionsrechnung}{Formeln für Investitionsrechnung}{}{17cm}

\clearpage
\subsubsection{Kostenvergleichsrechnung}
Vergleich der Gesamtkosten zwischen verschiedenen Investitionen.\\
Eignet sich, wenn Nutzen der Varianten etwa gleich oder unbekannt.

\paragraph{Idee:} Die Kostenvergleichsrechnung ziel nur darauf ab die Kosten eines Projektes zu vergleichen

\paragraph{Auswahlkriterium:} Gewählt wird das Projekt mit den niedrigeren Kosten

\paragraph{Kritik:}
\begin{compactitem}
	\item Reines Vergleichsverfahren
	\item Keinen Zeitwert des Geldes
	\item Hohe Anforderungen an Vorbedingungen
\end{compactitem}

\paragraph{Jährliche Gesamtkosten:} $\text{Jährliche Betriebskosten} + \text{Jährliche Abschreibungen} + \text{Zinskosten}$

\clearpage
\subsubsection{Gewinnvergleichsmethode}
Vergleich des Gewinns mehrerer Investitionen.\\
Eignet sich, wenn Kapitaleinsatz und Nutzungsdauer etwa gleich.

\paragraph{Idee:} Reiner Vergleich der durchschnittlichen Erträge von zwei Projekten

\paragraph{Auswahlkriterium:}
\begin{compactitem}
	\item Projekt muss Gewinn erwirtschaften
	\item Gewählt wird das Projekt mit dem höheren Ertrag
\end{compactitem}

\paragraph{Kritik:} Keine Berücksichtigung des Kapitaleinsatzes

\paragraph{Jährlicher Gewinn:} $\text{Jährlicher Erlös} - \text{Jährliche Betriebskosten} - \text{Jährliche Abschreibungen} - \text{Zinskosten}$\\
$\Rightarrow$ $\text{Jährlicher Erlös} - \text{Jährliche Gesamtkosten}$

\clearpage
\subsubsection{Renditeberechnung ROI}
Berechnen der Bruttorendite der Investitionen in Prozent des $\varnothing$ Kapitaleinsatzes.\\
Eignet sich zur Beurteilung einzelner Investitionen oder zum Vergleich unterschiedlicher Investitionsvarianten (die nicht direkt miteinander verglichen werden können).

\paragraph{ROI [\%]:} $\dfrac{\text{Gewinn} + \text{Zinskosten}}{\varnothing \text{Kapitaleinsatz}}\cdot 100$

\clearpage
\subsubsection{Amortisationsrechnung (Pay-Back-Methode)}
Ermitteln der Zeit, die verstreicht, bis eine Investition durch ihre Rückflüsse (Cashflows) amortisiert ist.\\
Eignet sich für Überschlagsrechnungen und als Risikomaßstab.

\paragraph{Idee:} Berechnung des Zeitpunktes zu dem ein Projekt zurückgezahlt wurde

\paragraph{Bedingung:} Laufzeit vergleichbar
\paragraph{Methode:} Abziehen der Zuflüsse von den Investitionen des Projektes und Überprüfen, wann das Projekt positiv wird

\paragraph{Auswahlkriterium:}
\begin{compactitem}
	\item Ein Projekt wird gemacht, wenn es seine Investition zurückzahlt
	\item Gewählt wird das Projekt mit der schnelleren Rückzahlungszeit
\end{compactitem}

\paragraph{Kritik:}
\begin{compactitem}
	\item Keinen Zeitwert des Geldes
	\item Keine strategischen Entscheidungen
\end{compactitem}

\redbox{$\text{Wiedergewinnungszeit} = \dfrac{\text{Kapitaleinsatz}}{\text{Jährlicher Nutzen}}$}

\redbox{$\text{Rückflusszahl} = \dfrac{\text{Nutzungsdauer}}{\text{Wiedergewinnungszeit}}$}

\clearpage
\subsection{Dynamische Methoden}
\clearpage
\subsubsection{Interner Ertragssatz (IRR)}
\imgw{Bilder/IRR}{IRR}{}{16cm}

\clearpage
\subsubsection{Gegenwart-Methode (Kapitalwertmethode)}
\imgw{Bilder/GWM}{Gegenwart-Methode}{}{16cm}

\clearpage
\subsubsection{Annuitätenmethode (Kapitalwertmethode)}
\imgw{Bilder/AM}{Annuitätenmethode}{}{16cm}

\section{Abgeleitete Größen}
\section{Nominal und effektiv Zins}
\section{Skonto}
\section{Rentenbegriff}
\section{Barwert und Endwert}
\section{Rückverteilungsfaktoren}

% ###################################
% # Literaturverzeichnis mit BibTeX #
% ###################################

\setboolean{showLiteratur}{\varZeigeLiteraturverzeichnis}
\ifthenelse{\boolean{showLiteratur}}{
    \clearpage
    \bibliography{literatur}
    \bibliographystyle{\varLiteraturLayout}
}{}

% #######################
% # Ende des Dokumentes #
% #######################

\end{document}
