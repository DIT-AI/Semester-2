% ############################
% # Eigene Befehle einbinden #
% ############################

% Eigene Befehle eignen sich gut um Abkürzungen für lange Befehle zu erstellen. Die Syntax ist folgende:
% \newcommand{neuer Befahl}{ein langer Befehl}
% Das folgende Beispiel fügt ein Bild mit bestimmten vorgegebenen Optionen ein:
\newcommand{\cImage}[1]{
	\begin{figure}[h!]
		\centering
		\includegraphics[width=0.50\textwidth]{#1}
	\end{figure}
}

% Images
\newcommand{\imgplaceholder}{
	\begin{center}
		\begin{figure}[h!]
			\centering \fcolorbox{orange}{orange}{\white{Hier kommt noch ein Bild}}
			\caption{}
			\label{}
		\end{figure}
	\end{center}
}
\newcommand{\img}[3]{
	\begin{center}
		\begin{figure}[h!]
			\centering \includegraphics[width=\linewidth]{#1}
			\caption{#2}
			\label{#3}
		\end{figure}
	\end{center}
}
\newcommand{\imgh}[4]{
	\begin{center}
		\begin{figure}[h!]
			\centering \includegraphics[height=#4]{#1}
			\caption{#2}
			\label{#3}
		\end{figure}
	\end{center}
}
\newcommand{\imgw}[4]{
	\begin{center}
		\begin{figure}[h!]
			\centering \includegraphics[width=#4]{#1}
			\caption{#2}
			\label{#3}
		\end{figure}
	\end{center}
}
\newcommand{\imgwh}[5]{
	\begin{center}
		\begin{figure}[h!]
			\centering \includegraphics[width=#4,height=#5]{#1}
			\caption{#2}
			\label{#3}
		\end{figure}
	\end{center}
}

% #1 ist dabei ein Parameter, den man \cImage übergeben muss. In 10_Titelseite.tex wird dieser Befehl verwendet. Der Parameter ist dort Bilder/titelseite.jpg.
% Benötigt man keine Parameter, dann lässt man [1] weg. Werden zusätzliche Parameter benötigt, dann kann man die Zahl auf maximal 9 erhöhen.

% Disjunkte Vereinigung
\makeatletter
\def\moverlay{\mathpalette\mov@rlay}
\def\mov@rlay#1#2{\leavevmode\vtop{%
		\baselineskip\z@skip \lineskiplimit-\maxdimen
		\ialign{\hfil$\m@th#1##$\hfil\cr#2\crcr}}}
\newcommand{\charfusion}[3][\mathord]{
	#1{\ifx#1\mathop\vphantom{#2}\fi
		\mathpalette\mov@rlay{#2\cr#3}
	}
	\ifx#1\mathop\expandafter\displaylimits\fi}
\makeatother

\newcommand{\cupdot}{\charfusion[\mathbin]{\cup}{\cdot}}
\newcommand{\bigcupdot}{\charfusion[\mathop]{\bigcup}{\cdot}}
\newcommand{\bigX}{\charfusion[\mathop]{{\LARGE\text{$\times$}}}{\ }}

% circled text
\newcommand{\circled}[1]{\tikz[baseline=(char.base)]{
		\node[shape=circle,draw,inner sep=2pt] (char) {#1};}}
% go high
\newcommand{\gohigh}{\tikz{
		\draw (0,0) -- (0.5em,0) -- (0.5em,1em) -- (1em,1em);}}
% go low
\newcommand{\golow}{\tikz{
			\draw (0,1em) -- (0.5em,1em) -- (0.5em,0) -- (1em,0);}}

\newcommand{\tx}[1]{\text{#1}}

% Text Color
\definecolor{mygreen}{rgb}{0.1, 0.65, 0.1}
\definecolor{amethyst}{rgb}{0.6, 0.4, 0.8}
\newcommand{\orange}[1]{\textcolor{orange}{#1}}
\newcommand{\red}[1]{\textcolor{red}{#1}}
\newcommand{\blue}[1]{\textcolor{blue}{#1}}
\newcommand{\green}[1]{\textcolor{mygreen}{#1}}
\newcommand{\violett}[1]{\textcolor{amethyst}{#1}}
\newcommand{\gray}[1]{\textcolor{gray}{#1}}
\newcommand{\brown}[1]{\textcolor{brown}{#1}}
\newcommand{\black}[1]{\textcolor{black}{#1}}
\newcommand{\white}[1]{\textcolor{white!0}{#1}}

% Underline Color
\newcommand{\orangeul}[1]{\colorlet{saved}{.}\orange{\ul{\color{saved}#1}}}
\newcommand{\redul}[1]{\colorlet{saved}{.}\red{\ul{\color{saved}#1}}}
\newcommand{\blueul}[1]{\colorlet{saved}{.}\blue{\ul{\color{saved}#1}}}
\newcommand{\greenul}[1]{\colorlet{saved}{.}\green{\ul{\color{saved}#1}}}
\newcommand{\violettul}[1]{\colorlet{saved}{.}\violett{\ul{\color{saved}#1}}}
\newcommand{\grayul}[1]{\colorlet{saved}{.}\gray{\ul{\color{saved}#1}}}
\newcommand{\brownul}[1]{\colorlet{saved}{.}\brown{\ul{\color{saved}#1}}}
\newcommand{\blackul}[1]{\colorlet{saved}{.}\black{\ul{\color{saved}#1}}}
\newcommand{\whiteul}[1]{\colorlet{saved}{.}\white{\ul{\color{saved}#1}}}

% Underbrace Color
\newcommand{\redunderbrace}[2]{\colorlet{saved}{.}\red{\underbrace{\color{saved}#1}_{#2}}}
\newcommand{\blueunderbrace}[2]{\colorlet{saved}{.}\blue{\underbrace{\color{saved}#1}_{#2}}}
\newcommand{\greenunderbrace}[2]{\colorlet{saved}{.}\green{\underbrace{\color{saved}#1}_{#2}}}
\newcommand{\grayunderbrace}[2]{\colorlet{saved}{.}\gray{\underbrace{\color{saved}#1}_{#2}}}

% Overbrace Color
\newcommand{\redoverbrace}[2]{\colorlet{saved}{.}\red{\overbrace{\color{saved}#1}_{#2}}}
\newcommand{\blueoverbrace}[2]{\colorlet{saved}{.}\blue{\overbrace{\color{saved}#1}_{#2}}}
\newcommand{\greenoverbrace}[2]{\colorlet{saved}{.}\green{\overbrace{\color{saved}#1}_{#2}}}
\newcommand{\grayoverbrace}[2]{\colorlet{saved}{.}\gray{\overbrace{\color{saved}#1}_{#2}}}

% Under
\newcommand{\under}[1]{\colorlet{saved}{.}\white{\underbrace{\black{#1}}}}
\newcommand{\redunder}[2]{\colorlet{saved}{.}\white{\underbrace{\black{#1}}_{\red{#2}}}}
\newcommand{\blueunder}[2]{\colorlet{saved}{.}\white{\underbrace{\black{#1}}_{\blue{#2}}}}
\newcommand{\greenunder}[2]{\colorlet{saved}{.}\white{\underbrace{\black{#1}}_{\green{#2}}}}
\newcommand{\grayunder}[2]{\colorlet{saved}{.}\white{\underbrace{\black{#1}}_{\gray{#2}}}}

% Over
\renewcommand{\over}[1]{\colorlet{saved}{.}\white{\overbrace{\black{#1}}}}
\newcommand{\redover}[2]{\colorlet{saved}{.}\white{\overbrace{\black{#1}}_{\red{#2}}}}
\newcommand{\blueover}[2]{\colorlet{saved}{.}\white{\overbrace{\black{#1}}_{\blue{#2}}}}
\newcommand{\greenover}[2]{\colorlet{saved}{.}\white{\overbrace{\black{#1}}_{\green{#2}}}}
\newcommand{\grayover}[2]{\colorlet{saved}{.}\white{\overbrace{\black{#1}}_{\gray{#2}}}}

% cBoxes
\newcommand{\cbox}[1]{{\vspace{-0.4cm}\begin{center}#1\end{center}\vspace{-0.4cm}}}
\newcommand{\graycbox}[1]{{\vspace{-0.4cm}\begin{center}\vspace{2pt}\fcolorbox{black!80}{black!10}{#1}\vspace{2pt}\end{center}\vspace{-0.4cm}}}
\newcommand{\redcbox}[1]{{\vspace{-0.4cm}\begin{center}\vspace{2pt}\fcolorbox{red}{red!15}{#1}\vspace{2pt}\end{center}\vspace{-0.4cm}}}
\newcommand{\greencbox}[1]{{\vspace{-0.4cm}\begin{center}\vspace{2pt}\fcolorbox{green}{green!15}{#1}\vspace{2pt}\end{center}\vspace{-0.4cm}}}
\newcommand{\bluecbox}[1]{{\vspace{-0.4cm}\begin{center}\vspace{2pt}\fcolorbox{blue}{blue!15}{#1}\vspace{2pt}\end{center}\vspace{-0.4cm}}}

% Boxes
\newcommand{\graybox}[1]{{\vspace{2pt}\fcolorbox{black!80}{black!10}{#1}\vspace{2pt}}}
\newcommand{\redbox}[1]{{\vspace{2pt}\fcolorbox{red}{red!15}{#1}\vspace{2pt}}}
\newcommand{\greenbox}[1]{{\vspace{2pt}\fcolorbox{green}{green!15}{#1}\vspace{2pt}}}
\newcommand{\bluebox}[1]{{\vspace{2pt}\fcolorbox{blue}{blue!15}{#1}\vspace{2pt}}}
\newcommand{\Graybox}[1]{\fcolorbox{black!80}{black!10}{\parbox{\dimexpr\linewidth-2\fboxsep-2\fboxrule\relax}{#1}}}
\newcommand{\Redbox}[1]{\fcolorbox{red}{red!15}{\parbox{\dimexpr\linewidth-2\fboxsep-2\fboxrule\relax}{#1}}}
\newcommand{\Greenbox}[1]{\fcolorbox{green}{green!15}{\parbox{\dimexpr\linewidth-2\fboxsep-2\fboxrule\relax}{#1}}}
\newcommand{\Bluebox}[1]{\fcolorbox{blue}{blue!15}{\parbox{\dimexpr\linewidth-2\fboxsep-2\fboxrule\relax}{#1}}}
% Frames
\newcommand{\grayframe}[1]{{\color{black!30}\fbox{\color{black}#1}}}
\newcommand{\redframe}[1]{{\color{red}\fbox{\color{black}#1}}}
\newcommand{\greenframe}[1]{{\color{green}\fbox{\color{black}#1}}}
\newcommand{\blueframe}[1]{{\color{blue}\fbox{\color{black}#1}}}
\newcommand{\Grayframe}[1]{{\color{black!30}\fbox{\color{black}\parbox{\dimexpr\linewidth-2\fboxsep-2\fboxrule\relax}{#1}}}}
\newcommand{\Redframe}[1]{{\color{red}\fbox{\color{black}\parbox{\dimexpr\linewidth-2\fboxsep-2\fboxrule\relax}{#1}}}}
\newcommand{\Greenframe}[1]{{\color{green}\fbox{\color{black}\parbox{\dimexpr\linewidth-2\fboxsep-2\fboxrule\relax}{#1}}}}
\newcommand{\Blueframe}[1]{{\color{blue}\fbox{\color{black}\parbox{\dimexpr\linewidth-2\fboxsep-2\fboxrule\relax}{#1}}}}

% Bsp
%\grayframe{grayframe}
%\redframe{redframe}
%\greenframe{greenframe}
%\Grayframe{Grayframe}
%\Redframe{Redframe}
%\Greenframe{Greenframe}
%\graybox{graybox}
%\redbox{redbox}
%\greenbox{greenbox}
%\Graybox{Graybox}
%\Redbox{Redbox}
%\Greenbox{Greenbox}



% Usefull
\renewcommand{\emph}[1]{\textit{#1}}

\newtheoremstyle{satzstyle}{15pt}{15pt}{}{-42pt}{\bf}{:}{1em}{}
\theoremstyle{satzstyle}
\newtheorem*{satz}{Satz}
\newtheoremstyle{korollarstyle}{15pt}{15pt}{}{-66pt}{\bf}{:}{1em}{}
\theoremstyle{korollarstyle}
\newtheorem*{korollar}{Korollar}
\newtheoremstyle{theoremstyle}{15pt}{15pt}{}{-67pt}{\bf}{:}{1em}{}
\theoremstyle{theoremstyle}
\newtheorem*{theorem}{Theorem}
\newtheoremstyle{lemmastyle}{15pt}{15pt}{}{-59pt}{\bf}{:}{1em}{}
\theoremstyle{lemmastyle}
\newtheorem*{lemma}{Lemma}
\newcommand{\pp}{\marginpar[\red{\bf (P)}]{\red{\bf (P)}}}
\newcommand{\PP}{\red{\bf Prüfungsrelevant (P)\quad}\marginpar[\red{\bf (P)}]{\red{\bf (P)}}}
\newcommand{\Def}{\paragraph{Definition:}}
\newcommand{\Bem}{\paragraph{Bemerkung:}}
\newcommand{\Bsp}{\paragraph{Beispiel:}}
\newcommand{\Bsps}{\paragraph{Beispiele:}}
\newcommand{\Beachte}{\paragraph{Beachte:}}
\newcommand{\Beweis}{\paragraph{Beweis:}}
\newcommand{\Formel}{\paragraph{Formel:}}
\newcommand{\Problem}{\paragraph{Problem:}}
\newcommand{\Beh}{\paragraph{Behauptung:}}
\newcommand{\Prosa}{\paragraph{Prosa:}}
\newcommand{\Frage}{\paragraph{Frage:}}
\newcommand{\Idee}{\paragraph{Idee:}}
\newcommand{\Allg}{\paragraph{Allgemein:}}
\newcommand{\Trick}{\paragraph{Trick:}}
\newcommand{\Tipp}{\paragraph{Tipp:}}
\newcommand{\Merke}{\paragraph{Merke:}}
\newcommand{\Ziel}{\paragraph{Ziel:}}
\newcommand{\Praxis}{\paragraph{Praxis:}}
%\newcommand{\Satz}[1]{\paragraph{Satz:}\graybox{#1\hfill}}
%\newcommand{\Satz}[1]{\paragraph{Satz:}
%	
%	\Graybox{#1}}
\newcommand{\Satz}[1]{\begin{satz}\begin{minipage}{\linewidth}\Graybox{#1}\end{minipage}\end{satz}}
%\newcommand{\SATZ}[1]{\paragraph{Satz:}
%	
%	\Graybox{#1}}
\newcommand{\Korollar}[1]{\begin{korollar}\Graybox{#1}\end{korollar}}
%\newcommand{\Korollar}[1]{\paragraph{Korollar:}
%	
%	\Graybox{#1}}
\newcommand{\Theorem}[1]{\begin{theorem}\Graybox{#1}\end{theorem}}
%\newcommand{\Theorem}[1]{\paragraph{Theorem:}
%	
%	\Graybox{#1}}
\newcommand{\Lemma}[1]{\begin{lemma}\Graybox{#1}\end{lemma}}
%\newcommand{\Lemma}[1]{\paragraph{Lemma:}
%	
%	\Graybox{#1}}

% Math
\newcommand{\N}{\mathbb{N}}
\newcommand{\No}{\mathbb{N}_0}
\newcommand{\Z}{\mathbb{Z}}
\newcommand{\Zp}{\mathbb{Z}^+}
\newcommand{\Zop}{\mathbb{Z}_0^+}
\newcommand{\Zm}{\mathbb{Z}^-}
\newcommand{\Zom}{\mathbb{Z}_0^-}
\newcommand{\Q}{\mathbb{Q}}
\newcommand{\Qp}{\mathbb{Q}^+}
\newcommand{\Qop}{\mathbb{Q}_0^+}
\newcommand{\Qm}{\mathbb{Q}^-}
\newcommand{\Qom}{\mathbb{Q}_0^-}
\newcommand{\R}{\mathbb{R}}
\newcommand{\Rp}{\mathbb{R}^+}
\newcommand{\Rop}{\mathbb{R}_0^+}
\newcommand{\Rm}{\mathbb{R}^-}
\newcommand{\Rom}{\mathbb{R}_0^-}
\newcommand{\C}{\mathbb{C}}
\newcommand{\F}{\mathbb{F}}
\newcommand{\G}{\mathbb{G}}
\newcommand{\D}{\mathbb{D}}
\renewcommand{\L}{\mathbb{L}}

% Grad
\newcommand{\degree}{{^\circ}}

% Differenzial
\newcommand{\diff}[1]{\mathrm{d}#1}
\newcommand{\df}[2]{\dfrac{\diff{#1}}{\diff{#2}}}
\newcommand{\ddf}[2]{\dfrac{\diff{#1}}{\diff{#2}^2}}
% Integral d korrekt formatieren
\newcommand{\intd}[1]{\ \diff{#1}}
% ^-1
\newcommand{\inv}[1]{{#1}^{-1}}
% Entspricht
\newcommand{\entspricht}{\stackrel{\wedge}{=}}
% Lines
\newcommand{\oline}[1]{\overline{#1}}
\newcommand{\ol}[1]{\overline{#1}}
%\newcommand{\uline}[1]{\underline{#1}}
\newcommand{\ul}[1]{\underline{#1}}
% Displaystyle
\newcommand{\ds}{\displaystyle}
% i
\renewcommand{\i}{\imath}
\newcommand{\ii}{\imath^2}
% Arg()
\newcommand{\Arg}{\mathbin\text{Arg}}
% Ln()
\newcommand{\Ln}{\mathbin\text{Ln}}
% GF()
\newcommand{\GF}{\mathbin\text{GF}}
% ggT(), kgV()
\newcommand{\ggT}{\mathbin\text{ggT}}
\newcommand{\kgV}{\mathbin\text{kgV}}
% sgn()
\newcommand{\sgn}{\mathbin\text{sgn}}
% grad()
\newcommand{\grad}{\mathbin\text{grad}}
%
\newcommand{\arccot}{\mathbin\text{arccot}}
\newcommand{\arsinh}{\mathbin\text{arsinh}}
\newcommand{\arcosh}{\mathbin\text{arcosh}}
\newcommand{\artanh}{\mathbin\text{artanh}}
\newcommand{\arcoth}{\mathbin\text{arcoth}}
% impliziert, äquivalent
\newcommand{\imp}{\rightarrow}
\newcommand{\eq}{\leftrightarrow}

%
\newcommand{\lxor}{\oplus}

% Vectoren und Matrizen
\newcommand{\vv}[2]{$\begin{pmatrix}#1\\#2\end{pmatrix}$}
\newcommand{\vvv}[3]{$\begin{pmatrix}#1\\#2\\#3\end{pmatrix}$}
\newcommand{\vvvv}[4]{$\begin{pmatrix}#1\\#2\\#3\\#4\end{pmatrix}$}

\newcommand{\vvs}[2]{$\begin{pmatrix}#1&#2\end{pmatrix}$}
\newcommand{\vvvs}[3]{$\begin{pmatrix}#1&#2&#3\end{pmatrix}$}
\newcommand{\vvvvs}[4]{$\begin{pmatrix}#1&#2&#3&#4\end{pmatrix}$}

\newcommand{\mm}[4]{$\begin{pmatrix}#1&#2\\#3&#4\end{pmatrix}$}
\newcommand{\mmm}[9]{$\begin{pmatrix}#1&#2&#3\\#4&#5&#6\\#7&#8&#9\end{pmatrix}$}

\newcommand{\mmmm}[8]{
	\def\Argi{{#1}}%
	\def\Argii{{#2}}%
	\def\Argiii{#3}%
	\def\Argiv{#4}%
	\def\Argv{#5}%
	\def\Argvi{#6}%
	\def\Argvii{#7}%
	\def\Argviii{#8}%
	\mmmmContinue
}

\newcommand{\mmmmContinue}[8]{$\begin{pmatrix}\Argi&\Argii&\Argiii&\Argiv\\\Argv&\Argvi&\Argvii&\Argviii\\#1&#2&#3&#4\\#5&#6&#7&#8\end{pmatrix}$}
