% ####################
% # Pakete einbinden #
% ####################

% Pakete erweitern LaTeX um zusätzliche Funktionen. Dies ist eine Satz nützlicher Pakete.
% Weitere sollten in der Datei"`01_EigenePakete.tex"' hinzugefügt werden.
\usepackage[utf8x]{inputenc}                                                  % Legt die Zeichenkodierung fest, z.B UTF8
\usepackage[T1]{fontenc}                                                      % Verwendung der Zeichentabelle T1, für deutschsprachige Dokumente sinnvoll
\usepackage[ngerman,english]{babel}                                           % Silbentrennung nach neuer deutscher und englischer Rechtschreibung
\usepackage{amsmath}                                                          % Mathepaket
\usepackage{amsfonts}
\usepackage{amssymb}                                                          % Mathepaket
\usepackage{xifthen}                                                          % Wird benötigt um \ifthenelse zu benutzen
%\usepackage{graphicx}                                                         % Zum flexiblen Einbinden von Grafiken, pdftex ist optional
\usepackage[pdftex]{graphicx}                                                 % Zum flexiblen Einbinden von Grafiken, pdftex ist optional
\usepackage{units}                                                            % Ermöglicht die Nutzung von \unit[Zahl]{Einheit}
\usepackage{setspace}                                                         % Einfaches wechseln zwischen unterschiedlichen Zeilenabständen
%\usepackage{hyperref}                                                         % Verlinkt Textstellen im PDF Dokument
\usepackage[pdfpagelabels]{hyperref}                                          % Verlinkt Textstellen im PDF Dokument
\usepackage[font=small,labelfont=bf,labelsep=endash,format=plain]{caption}    % Darstellung für Caption s.u.
\usepackage{subfig}                                                           % Bilder nebeneinander
\usepackage{wrapfig}                                                          % Fließtext um Figure-Umgebung
\usepackage{cite}                                                             % Zusatzfunktionen zum zitieren
\usepackage{scrpage2}                                                         % Wird für Kopf- und Fußzeile benötigt
\usepackage{array,dcolumn}                                                    % Beide Pakete werden für die Ausrichtung der Tabellenspalten benötigt


% ############################
% # weitere Pakete einbinden #
% ############################

% \usepackage{showframe}
% Die Folgenden Pakete sind schon eingebunden (siehe 00_Protokoll.tex):
% \usepackage[utf8x]{inputenc}                             % Legt die Zeichenkodierung fest, z.B UTF8
% \usepackage[T1]{fontenc}                                 % Verwendung der Zeichentabelle T1, für deutschsprachige Dokumente sinnvoll
% \usepackage[ngerman,english]{babel}                      % Silbentrennung nach neuer deutscher und englischer Rechtschreibung
% \usepackage{amsmath}                                     % Mathepaket
% \usepackage{amssymb}                                     % Mathepaket
% \usepackage{ifthenx}                                     % Wird benötigt um \ifthenelse zu benutzen
% \usepackage[pdftex]{graphicx}                            % Zum flexiblen Einbinden von Grafiken, pdftex ist optional
% \usepackage{rotating}                            % Zum drehen von Objekten, pdftex ist optional
  \usepackage[pdftex]{rotating}                            % Zum drehen von Objekten, pdftex ist optional
% \usepackage{units}                                       % Ermöglicht die Nutzung von \unit[Zahl]{Einheit}
% \usepackage{setspace}                                    % Einfaches wechseln zwischen unterschiedlichen Zeilenabständen
% \usepackage[pdfpagelabels]{hyperref}                     % Verlinkt Textstellen im PDF Dokument
% \usepackage[font=small,labelfont=bf,labelsep=endash,format=plain]{caption}
%                                                          % Darstellung für Caption s.u.
% \usepackage{subfig}                                      % Bilder nebeneinander
% \usepackage{wrapfig}                                     % Fließtext um Figure-Umgebung
% \usepackage{cite}                                        % Zusatzfunktionen zum zitieren
% \usepackage{scrpage2}                                    % Wird für Kopf- und Fußzeile benötigt
% \usepackage{array,dcolumn}                               % Beide Pakete werden für die Ausrichtung der Tabellenspalten benötigt
  \usepackage{enumerate}                                   % Um andere Aufzählungsvarianten zu erzeugen http://ctan.org/pkg/enumerate
  \usepackage{xcolor}
  \usepackage{ulem}
% \usepackage{mathtools}
  \usepackage{longtable}
  \usepackage{tabularx}                                    % http://ctan.org/pkg/tabularx
  \usepackage{booktabs}                                    % http://ctan.org/pkg/booktabs
% \usepackage{a4wide}
  \usepackage{geometry}
  \usepackage{amsthm}
% \usepackage{pstricks-add}
% \usepackage{pstricks}
  \usepackage{pgf,tikz}
  \usetikzlibrary{arrows}
  \usepackage{chngcntr}
  \usepackage{trfsigns}
  \usepackage{multicol}
  \usepackage{eurosym}
  \usepackage{pdfpages}
  \usepackage[nomessages]{fp}% http://ctan.org/pkg/fp
  \usepackage{calc}
  \usepackage{listings}
  \usepackage{paralist}
\input{kvmacros}

% Spezialpakete
\usepackage{fp}
\usepackage{tikz}
\usepackage{xcolor}
% TikZ-Bibliotheken
\usetikzlibrary{arrows}
\usetikzlibrary{shapes}
\usetikzlibrary{decorations.pathmorphing}
\usetikzlibrary{decorations.pathreplacing}
\usetikzlibrary{decorations.shapes}
\usetikzlibrary{decorations.text}
\usetikzlibrary{calc}
\usetikzlibrary{snakes}
\usetikzlibrary{positioning}
